\documentclass[12pt, titlepage]{article}

\usepackage{booktabs}
\usepackage{tabularx}
\usepackage{hyperref}
\usepackage{float}
\hypersetup{
    colorlinks,
    citecolor=blue,
    filecolor=black,
    linkcolor=red,
    urlcolor=blue
}
\usepackage[round]{natbib}

%% Comments

\usepackage{color}

\newif\ifcomments\commentstrue %displays comments
%\newif\ifcomments\commentsfalse %so that comments do not display

\ifcomments
\newcommand{\authornote}[3]{\textcolor{#1}{[#3 ---#2]}}
\newcommand{\todo}[1]{\textcolor{red}{[TODO: #1]}}
\else
\newcommand{\authornote}[3]{}
\newcommand{\todo}[1]{}
\fi

\newcommand{\wss}[1]{\authornote{magenta}{SS}{#1}} 
\newcommand{\plt}[1]{\authornote{cyan}{TPLT}{#1}} %For explanation of the template
\newcommand{\an}[1]{\authornote{cyan}{Author}{#1}}

%% Common Parts

\newcommand{\progname}{Software Engineering} % PUT YOUR PROGRAM NAME HERE
\newcommand{\teamname}{EvENGage}
\newcommand{\authname}{Team 4, \teamname
\\ Virochaan Ravichandran Gowri
\\ Omar Al-Asfar
\\ Rayyan Suhail
\\ Ibrahim Quraishi
\\ Mohammad Mahdi Mahboob} % AUTHOR NAMES

\newcommand{\prjdesc}{MES Event Management Registration, Administration, and Survey Analytics}

\usepackage{hyperref}
    \hypersetup{colorlinks=true, linkcolor=blue, citecolor=blue, filecolor=blue,
                urlcolor=blue, unicode=false}
    \urlstyle{same}



\begin{document}

\title{System Verification and Validation Plan for \progname{}} 
\author{\authname}
\date{\today}
	
\maketitle

\pagenumbering{roman}

\section*{Revision History}

\begin{tabularx}{\textwidth}{p{3cm}p{2cm}X}
\toprule {\bf Date} & {\bf Version} & {\bf Notes}\\
\midrule
Date 1 & 1.0 & Notes\\
Date 2 & 1.1 & Notes\\
\bottomrule
\end{tabularx}

~\\
\wss{The intention of the VnV plan is to increase confidence in the software.
However, this does not mean listing every verification and validation technique
that has ever been devised.  The VnV plan should also be a \textbf{feasible}
plan. Execution of the plan should be possible with the time and team available.
If the full plan cannot be completed during the time available, it can either be
modified to ``fake it'', or a better solution is to add a section describing
what work has been completed and what work is still planned for the future.}

\wss{The VnV plan is typically started after the requirements stage, but before
the design stage.  This means that the sections related to unit testing cannot
initially be completed.  The sections will be filled in after the design stage
is complete.  the final version of the VnV plan should have all sections filled
in.}

\newpage

\tableofcontents

\listoftables
\wss{Remove this section if it isn't needed}

\listoffigures
\wss{Remove this section if it isn't needed}

\newpage

\section{Symbols, Abbreviations, and Acronyms}

\renewcommand{\arraystretch}{1.2}
\begin{tabular}{l l} 
  \toprule		
  \textbf{symbol} & \textbf{description}\\
  \midrule 
  T & Test\\
  \bottomrule
\end{tabular}\\

\wss{symbols, abbreviations, or acronyms --- you can simply reference the SRS
  \citep{SRS} tables, if appropriate}

\wss{Remove this section if it isn't needed}

\newpage

\pagenumbering{arabic}

This document ... \wss{provide an introductory blurb and roadmap of the
  Verification and Validation plan}

\section{General Information}

\subsection{Summary}

\wss{Say what software is being tested.  Give its name and a brief overview of
  its general functions.}

\subsection{Objectives}

\wss{State what is intended to be accomplished.  The objective will be around
  the qualities that are most important for your project.  You might have
  something like: ``build confidence in the software correctness,''
  ``demonstrate adequate usability.'' etc.  You won't list all of the qualities,
  just those that are most important.}

\wss{You should also list the objectives that are out of scope.  You don't have 
the resources to do everything, so what will you be leaving out.  For instance, 
if you are not going to verify the quality of usability, state this.  It is also 
worthwhile to justify why the objectives are left out.}

\wss{The objectives are important because they highlight that you are aware of 
limitations in your resources for verification and validation.  You can't do everything, 
so what are you going to prioritize?  As an example, if your system depends on an 
external library, you can explicitly state that you will assume that external library 
has already been verified by its implementation team.}

\subsection{Challenge Level and Extras}

\wss{State the challenge level (advanced, general, basic) for your project.
Your challenge level should exactly match what is included in your problem
statement.  This should be the challenge level agreed on between you and the
course instructor.  You can use a pull request to update your challenge level
(in TeamComposition.csv or Repos.csv) if your plan changes as a result of the
VnV planning exercise.}

\wss{Summarize the extras (if any) that were tackled by this project.  Extras
can include usability testing, code walkthroughs, user documentation, formal
proof, GenderMag personas, Design Thinking, etc.  Extras should have already
been approved by the course instructor as included in your problem statement.
You can use a pull request to update your extras (in TeamComposition.csv or
Repos.csv) if your plan changes as a result of the VnV planning exercise.}

\subsection{Relevant Documentation}

\wss{Reference relevant documentation.  This will definitely include your SRS
  and your other project documents (design documents, like MG, MIS, etc).  You
  can include these even before they are written, since by the time the project
  is done, they will be written.  You can create BibTeX entries for your
  documents and within those entries include a hyperlink to the documents.}

\citet{SRS}

\wss{Don't just list the other documents.  You should explain why they are relevant and 
how they relate to your VnV efforts.}

\section{Plan}

\wss{Introduce this section.  You can provide a roadmap of the sections to
  come.}

\subsection{Verification and Validation Team}

\wss{Your teammates.  Maybe your supervisor.
  You should do more than list names.  You should say what each person's role is
  for the project's verification.  A table is a good way to summarize this information.}

\subsection{SRS Verification}

\wss{List any approaches you intend to use for SRS verification.  This may
  include ad hoc feedback from reviewers, like your classmates (like your
  primary reviewer), or you may plan for something more rigorous/systematic.}

\wss{If you have a supervisor for the project, you shouldn't just say they will
read over the SRS.  You should explain your structured approach to the review.
Will you have a meeting?  What will you present?  What questions will you ask?
Will you give them instructions for a task-based inspection?  Will you use your
issue tracker?}

\wss{Maybe create an SRS checklist?}

\subsection{Design Verification}

\wss{Plans for design verification}

\wss{The review will include reviews by your classmates}

\wss{Create a checklists?}

\subsection{Verification and Validation Plan Verification}

\wss{The verification and validation plan is an artifact that should also be
verified.  Techniques for this include review and mutation testing.}

\wss{The review will include reviews by your classmates}

\wss{Create a checklists?}

\subsection{Implementation Verification}

\wss{You should at least point to the tests listed in this document and the unit
  testing plan.}

\wss{In this section you would also give any details of any plans for static
  verification of the implementation.  Potential techniques include code
  walkthroughs, code inspection, static analyzers, etc.}

\wss{The final class presentation in CAS 741 could be used as a code
walkthrough.  There is also a possibility of using the final presentation (in
CAS741) for a partial usability survey.}

\subsection{Automated Testing and Verification Tools}

\begin{itemize}
    \item \textbf{Vitest:} Used as the primary unit and integration testing framework for the frontend. It integrates seamlessly with the Vite development environment and allows for efficient automated testing of components, hooks, and application logic. Vitest supports mocking and snapshot testing, which ensures that UI and logic remain consistent after code changes.

    \item \textbf{Playwright:} Utilized for automated end-to-end testing to simulate user interactions across different browsers and devices. This tool can be used to validate complete user workflows such as event registration and form creation and submissions. This will help ensure system functionality and reliability from the user's perspective.

    \item \textbf{Postman and Newman:} Employed for API-level testing and verification. Postman allows the team to design and execute test cases for RESTful endpoints, while Newman enables command-line execution of these tests within the CI/CD pipeline. Together, they ensure backend consistency and correct response handling.

    \item \textbf{GitHub Actions:} Serves as the continuous integration (CI) platform that automates testing and verification tasks. It will execute unit tests, end-to-end tests, and linters automatically on each push or pull request. This will ensure that all commits meet quality and functionality and don't break the system before merging.

    \item \textbf{pgTAP:} Used for database testing in PostgreSQL. It can help ensure database integrity and ensure that that database schema and back-end logic is implemented correctly through unit testing.
\end{itemize}

\subsection{Software Validation}

\wss{If there is any external data that can be used for validation, you should
  point to it here.  If there are no plans for validation, you should state that
  here.}

\wss{You might want to use review sessions with the stakeholder to check that
the requirements document captures the right requirements.  Maybe task based
inspection?}

\wss{For those capstone teams with an external supervisor, the Rev 0 demo should 
be used as an opportunity to validate the requirements.  You should plan on 
demonstrating your project to your supervisor shortly after the scheduled Rev 0 demo.  
The feedback from your supervisor will be very useful for improving your project.}

\wss{For teams without an external supervisor, user testing can serve the same purpose 
as a Rev 0 demo for the supervisor.}

\wss{This section might reference back to the SRS verification section.}

\section{System Tests}

This section outlines the the system wide tests to validate both functional and non-functional requirements defined in our Software Requirements Specification.

\subsection{Tests for Functional Requirements}

This section verifies that the implemented system fulfills all \textbf{Functional Requirements (FR-1 through FR-11)} defined in the SRS. 
The tests are grouped into three \textbf{Areas of Testing}: \textit{Admin Portal (AP)}, \textit{Attendee Application (AA)}, and \textit{System Integration and Security (SI)}. 
\subsubsection{Area of Testing 1 – Admin Portal}

\begin{enumerate}

\item[\textbf{Test-FR-AP-1}] \textbf{Create Form}\\
\textbf{Type:} Manual\\
\textbf{Initial State:} Admin is logged in with form creation privileges.\\
\textbf{Input/Condition:} Admin enters a new form name and adds multiple field types (text, checkbox, rating).\\
\textbf{Expected Output:} The new form module is created and is visible in the database and in the admin dashboard.\\
\textbf{Test Case Derivation:} This test verifies the system’s ability to support dynamic form creation and storage as required by FR-1 and FR-7.\\
\textbf{How the test will be performed:} The tester will create a new form and confirm it appears correctly in the interface and database.\\[6pt]

\item[\textbf{Test-FR-AP-2}] \textbf{View and Analyze Form Data}\\
\textbf{Type:} Manual\\
\textbf{Initial State:} At least two forms with collected responses exist.\\
\textbf{Input/Condition:} Admin selects multiple forms and clicks button to view report analytics\\
\textbf{Expected Output:} Aggregated analytics are displayed correctly and match backend data.\\
\textbf{Test Case Derivation:} This test confirms data organization and analytical visualization functionalities described in FR-2 and FR-8, ensuring accurate reporting and analytics.\\
\textbf{How the test will be performed:} The tester will generate combined analytics and confirm that the displayed charts match backend results.\\[6pt]

\item[\textbf{Test-FR-AP-3}] \textbf{Manage Events via Dashboard}\\
\textbf{Type:} Manual\\
\textbf{Initial State:} Several events with participant data exist.\\
\textbf{Input/Condition:} Admin accesses the dashboard and applies filters (e.g., pending payment).\\
\textbf{Expected Output:} Filtered results display accurately with correct status counts.\\
\textbf{Test Case Derivation:} This test validates event management, filtering, and data persistence mechanisms described in FR-5 and FR-6.\\
\textbf{How the test will be performed:} The tester will apply filters and verify that displayed data matches expected event records.\\[6pt]

\item[\textbf{Test-FR-AP-4}] \textbf{View and Export Event Analytics}\\
\textbf{Type:} Manual\\
\textbf{Initial State:} Events and surveys contain collected feedback.\\
\textbf{Input/Condition:} Admin selects an event and exports analytics.\\
\textbf{Expected Output:} Correct analytics display and downloadable report matches system data.\\
\textbf{Test Case Derivation:} This test ensures that analytics generation and report export satisfy the requirements for event statistics and reporting defined in FR-8 and FR-10.\\
\textbf{How the test will be performed:} The tester will view analytics for an event, export the report, and compare the output to dashboard data.\\[12pt]

\end{enumerate}

\subsubsection{Area of Testing 2 – Attendee Application}

\begin{enumerate}

\item[\textbf{Test-FR-AA-1}] \textbf{Register for Event}\\
\textbf{Type:} Automatic\\
\textbf{Initial State:} An event is open and attendee is authenticated.\\
\textbf{Input/Condition:} Attendee submits an event registration form.\\
\textbf{Expected Output:} Registration is saved and a confirmation message is displayed and sent to attendee and a QR ticket is generated.\\
\textbf{Test Case Derivation:} This test validates the registration workflow and QR ticket generation process as required by FR-3, FR-4, and FR-6.\\
\textbf{How the test will be performed:} An automated test will submit registration data and verify the confirmation message and QR code has been created. Will confirm that the event data has also been updated in the database.\\[6pt]

\item[\textbf{Test-FR-AA-2}] \textbf{Access Event Listings}\\
\textbf{Type:} Manual\\
\textbf{Initial State:} The system has multiple upcoming and past events.\\
\textbf{Input/Condition:} Attendee views ``Upcoming'' and ``Registered Events'' tabs.\\
\textbf{Expected Output:} Each tab displays correct events for that category.\\
\textbf{Test Case Derivation:} This test confirms accurate retrieval and filtering of events for authenticated users as required by FR-3 and ensures user can view event info.\\
\textbf{How the test will be performed:} The tester will view event lists and verify they match stored registration records. Tester will also ensure event info shows up correctly and matches database.\\[6pt]

\item[\textbf{Test-FR-AA-3}] \textbf{Fill out Form}\\
\textbf{Type:} Automatic\\
\textbf{Initial State:} A survey is available and attendee is authenticated.\\
\textbf{Input/Condition:} Attendee completes and submits the survey.\\
\textbf{Expected Output:} Survey responses are stored and a confirmation message appears.\\
\textbf{Test Case Derivation:} This test ensures that survey submission and feedback processing meet the specifications in FR-7, FR-9.\\
\textbf{How the test will be performed:} An automated test will submit survey responses and confirm that results are stored and submission is found in the centralized database.\\[12pt]

\end{enumerate}

\subsubsection{Area of Testing 3 – System Integration and Security}

\begin{enumerate}

\item[\textbf{Test-FR-SI-1}] \textbf{Real-Time Data Synchronization}\\
\textbf{Type:} Automatic\\
\textbf{Initial State:} Admin and attendee interfaces are active.\\
\textbf{Input/Condition:} Admin updates event details (e.g., venue or date).\\
\textbf{Expected Output:} Attendee view updates automatically without manual refresh.\\
\textbf{Test Case Derivation:} This test validates the real-time synchronization and data propagation features required by FR-6 and FR-7 and ensures that both databases are connected to the interface.\\
\textbf{How the test will be performed:} An automated test will modify event data and verify that the attendee view reflects the change immediately.\\[6pt]

\item[\textbf{Test-FR-SI-2}] \textbf{Role-Based Access Verification}\\
\textbf{Type:} Manual\\
\textbf{Initial State:} Admin and attendee accounts exist with distinct permissions.\\
\textbf{Input/Condition:} Each user attempts to access admin-only features.\\
\textbf{Expected Output:} Admin gains access but attendee is denied access with proper error handling.\\
\textbf{Test Case Derivation:} This test checks role-based access control and authorization mechanisms as defined in FR-11 ensuring only specific users have access to admin controls.\\
\textbf{How the test will be performed:} The tester will log in as both user types and attempt to access admin features to confirm proper restriction.\\[12pt]

\end{enumerate}

% \subsection{Functional Requirement Coverage Summary}




\subsection{Tests for Nonfunctional Requirements}

\wss{The nonfunctional requirements for accuracy will likely just reference the
  appropriate functional tests from above.  The test cases should mention
  reporting the relative error for these tests.  Not all projects will
  necessarily have nonfunctional requirements related to accuracy.}

\wss{For some nonfunctional tests, you won't be setting a target threshold for
passing the test, but rather describing the experiment you will do to measure
the quality for different inputs.  For instance, you could measure speed versus
the problem size.  The output of the test isn't pass/fail, but rather a summary
table or graph.}

\wss{Tests related to usability could include conducting a usability test and
  survey.  The survey will be in the Appendix.}

\wss{Static tests, review, inspections, and walkthroughs, will not follow the
format for the tests given below.}

\wss{If you introduce static tests in your plan, you need to provide details.
How will they be done?  In cases like code (or document) walkthroughs, who will
be involved? Be specific.}

\subsubsection{Area of Testing1}
		
\paragraph{Title for Test}

\begin{enumerate}

\item{test-id1\\}

Type: Functional, Dynamic, Manual, Static etc.
					
Initial State: 
					
Input/Condition: 
					
Output/Result: 
					
How test will be performed: 
					
\item{test-id2\\}

Type: Functional, Dynamic, Manual, Static etc.
					
Initial State: 
					
Input: 
					
Output: 
					
How test will be performed: 

\end{enumerate}

\subsubsection{Area of Testing2}

...

\subsection{Traceability Between Test Cases and Requirements}

\begin{table}[H]
\centering
\begin{tabular}{|l|l|}
\hline
\textbf{Test ID} & \textbf{Functional Requirements Covered} \\
\hline
Test-FR-AP-1 & FR-1, FR-7 \\
Test-FR-AP-2 & FR-2, FR-8 \\
Test-FR-AP-3 & FR-5, FR-6 \\
Test-FR-AP-4 & FR-8, FR-10 \\
Test-FR-AA-1 & FR-3, FR-4, FR-6 \\
Test-FR-AA-2 & FR-3\\
Test-FR-AA-3 & FR-7, FR-9\\
Test-FR-SI-1 & FR-6, FR-7 \\
Test-FR-SI-2 & FR-11 \\
\hline
\end{tabular}
\caption{Mapping of System Tests to Functional Requirements}
\end{table}

\section{Unit Test Description}

\wss{This section should not be filled in until after the MIS (detailed design
  document) has been completed.}

\wss{Reference your MIS (detailed design document) and explain your overall
philosophy for test case selection.}  

\wss{To save space and time, it may be an option to provide less detail in this section.  
For the unit tests you can potentially layout your testing strategy here.  That is, you 
can explain how tests will be selected for each module.  For instance, your test building 
approach could be test cases for each access program, including one test for normal behaviour 
and as many tests as needed for edge cases.  Rather than create the details of the input 
and output here, you could point to the unit testing code.  For this to work, you code 
needs to be well-documented, with meaningful names for all of the tests.}

\subsection{Unit Testing Scope}

\wss{What modules are outside of the scope.  If there are modules that are
  developed by someone else, then you would say here if you aren't planning on
  verifying them.  There may also be modules that are part of your software, but
  have a lower priority for verification than others.  If this is the case,
  explain your rationale for the ranking of module importance.}

\subsection{Tests for Functional Requirements}

\wss{Most of the verification will be through automated unit testing.  If
  appropriate specific modules can be verified by a non-testing based
  technique.  That can also be documented in this section.}

\subsubsection{Module 1}

\wss{Include a blurb here to explain why the subsections below cover the module.
  References to the MIS would be good.  You will want tests from a black box
  perspective and from a white box perspective.  Explain to the reader how the
  tests were selected.}

\begin{enumerate}

\item{test-id1\\}

Type: \wss{Functional, Dynamic, Manual, Automatic, Static etc. Most will
  be automatic}
					
Initial State: 
					
Input: 
					
Output: \wss{The expected result for the given inputs}

Test Case Derivation: \wss{Justify the expected value given in the Output field}

How test will be performed: 
					
\item{test-id2\\}

Type: \wss{Functional, Dynamic, Manual, Automatic, Static etc. Most will
  be automatic}
					
Initial State: 
					
Input: 
					
Output: \wss{The expected result for the given inputs}

Test Case Derivation: \wss{Justify the expected value given in the Output field}

How test will be performed: 

\item{...\\}
    
\end{enumerate}

\subsubsection{Module 2}

...

\subsection{Tests for Nonfunctional Requirements}

\wss{If there is a module that needs to be independently assessed for
  performance, those test cases can go here.  In some projects, planning for
  nonfunctional tests of units will not be that relevant.}

\wss{These tests may involve collecting performance data from previously
  mentioned functional tests.}

\subsubsection{Module ?}
		
\begin{enumerate}

\item{test-id1\\}

Type: \wss{Functional, Dynamic, Manual, Automatic, Static etc. Most will
  be automatic}
					
Initial State: 
					
Input/Condition: 
					
Output/Result: 
					
How test will be performed: 
					
\item{test-id2\\}

Type: Functional, Dynamic, Manual, Static etc.
					
Initial State: 
					
Input: 
					
Output: 
					
How test will be performed: 

\end{enumerate}

\subsubsection{Module ?}

...

\subsection{Traceability Between Test Cases and Modules}

\wss{Provide evidence that all of the modules have been considered.}
				
\bibliographystyle{plainnat}

\bibliography{../../refs/References}

\newpage

\section{Appendix}

This is where you can place additional information.

\subsection{Symbolic Parameters}

The definition of the test cases will call for SYMBOLIC\_CONSTANTS.
Their values are defined in this section for easy maintenance.

\subsection{Usability Survey Questions?}

\wss{This is a section that would be appropriate for some projects.}

\newpage{}
\section*{Appendix --- Reflection}

\wss{This section is not required for CAS 741}

The information in this section will be used to evaluate the team members on the
graduate attribute of Lifelong Learning.

The purpose of reflection questions is to give you a chance to assess your own
learning and that of your group as a whole, and to find ways to improve in the
future. Reflection is an important part of the learning process.  Reflection is
also an essential component of a successful software development process.  

Reflections are most interesting and useful when they're honest, even if the
stories they tell are imperfect. You will be marked based on your depth of
thought and analysis, and not based on the content of the reflections
themselves. Thus, for full marks we encourage you to answer openly and honestly
and to avoid simply writing ``what you think the evaluator wants to hear.''

Please answer the following questions.  Some questions can be answered on the
team level, but where appropriate, each team member should write their own
response:


\begin{enumerate}
  \item What went well while writing this deliverable? 
  \item What pain points did you experience during this deliverable, and how
    did you resolve them?
  \item What knowledge and skills will the team collectively need to acquire to
  successfully complete the verification and validation of your project?
  Examples of possible knowledge and skills include dynamic testing knowledge,
  static testing knowledge, specific tool usage, Valgrind etc.  You should look to
  identify at least one item for each team member.
  \item For each of the knowledge areas and skills identified in the previous
  question, what are at least two approaches to acquiring the knowledge or
  mastering the skill?  Of the identified approaches, which will each team
  member pursue, and why did they make this choice?
\end{enumerate}

\end{document}