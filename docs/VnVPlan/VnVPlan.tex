\documentclass[12pt, titlepage]{article}

\usepackage{booktabs}
\usepackage{tabularx}
\usepackage{hyperref}
\hypersetup{
    colorlinks,
    citecolor=blue,
    filecolor=black,
    linkcolor=red,
    urlcolor=blue
}
\usepackage[round]{natbib}
\usepackage{graphicx}
\usepackage{caption}
\usepackage{enumitem}
\usepackage{float}
\usepackage{longtable}
\usepackage[english]{babel}
\usepackage[acronym,nonumberlist]{glossaries}

%% Comments

\usepackage{color}

\newif\ifcomments\commentstrue %displays comments
%\newif\ifcomments\commentsfalse %so that comments do not display

\ifcomments
\newcommand{\authornote}[3]{\textcolor{#1}{[#3 ---#2]}}
\newcommand{\todo}[1]{\textcolor{red}{[TODO: #1]}}
\else
\newcommand{\authornote}[3]{}
\newcommand{\todo}[1]{}
\fi

\newcommand{\wss}[1]{\authornote{magenta}{SS}{#1}} 
\newcommand{\plt}[1]{\authornote{cyan}{TPLT}{#1}} %For explanation of the template
\newcommand{\an}[1]{\authornote{cyan}{Author}{#1}}

%% Common Parts

\newcommand{\progname}{Software Engineering} % PUT YOUR PROGRAM NAME HERE
\newcommand{\teamname}{EvENGage}
\newcommand{\authname}{Team 4, \teamname
\\ Virochaan Ravichandran Gowri
\\ Omar Al-Asfar
\\ Rayyan Suhail
\\ Ibrahim Quraishi
\\ Mohammad Mahdi Mahboob} % AUTHOR NAMES

\newcommand{\prjdesc}{MES Event Management Registration, Administration, and Survey Analytics}

\usepackage{hyperref}
    \hypersetup{colorlinks=true, linkcolor=blue, citecolor=blue, filecolor=blue,
                urlcolor=blue, unicode=false}
    \urlstyle{same}



\begin{document}

\title{System Verification and Validation Plan for \progname{}} 
\author{\authname}
\date{\today}
	
\maketitle

\pagenumbering{roman}

\section*{Revision History}

\begin{tabularx}{\textwidth}{p{3cm}p{2cm}X}
\toprule {\bf Date} & {\bf Version} & {\bf Notes}\\
\midrule
Date 1 & 1.0 & Notes\\
Date 2 & 1.1 & Notes\\
\bottomrule
\end{tabularx}

~\\
\wss{The intention of the VnV plan is to increase confidence in the software.
However, this does not mean listing every verification and validation technique
that has ever been devised.  The VnV plan should also be a \textbf{feasible}
plan. Execution of the plan should be possible with the time and team available.
If the full plan cannot be completed during the time available, it can either be
modified to ``fake it'', or a better solution is to add a section describing
what work has been completed and what work is still planned for the future.}

\wss{The VnV plan is typically started after the requirements stage, but before
the design stage.  This means that the sections related to unit testing cannot
initially be completed.  The sections will be filled in after the design stage
is complete.  the final version of the VnV plan should have all sections filled
in.}

\newpage

\tableofcontents

\listoftables
\wss{Remove this section if it isn't needed}

\listoffigures
\wss{Remove this section if it isn't needed}

\newpage

\section{Symbols, Abbreviations, and Acronyms}

\renewcommand{\arraystretch}{1.2}
\begin{tabular}{l l} 
  \toprule		
  \textbf{symbol} & \textbf{description}\\
  \midrule 
  T & Test\\
  \bottomrule
\end{tabular}\\

\wss{symbols, abbreviations, or acronyms --- you can simply reference the SRS
  \citep{SRS} tables, if appropriate}

\wss{Remove this section if it isn't needed}

\newpage

\pagenumbering{arabic}

This document ... \wss{provide an introductory blurb and roadmap of the
  Verification and Validation plan}

\section{General Information}

\subsection{Summary}

\wss{Say what software is being tested.  Give its name and a brief overview of
  its general functions.}

\subsection{Objectives}

\wss{State what is intended to be accomplished.  The objective will be around
  the qualities that are most important for your project.  You might have
  something like: ``build confidence in the software correctness,''
  ``demonstrate adequate usability.'' etc.  You won't list all of the qualities,
  just those that are most important.}

\wss{You should also list the objectives that are out of scope.  You don't have 
the resources to do everything, so what will you be leaving out.  For instance, 
if you are not going to verify the quality of usability, state this.  It is also 
worthwhile to justify why the objectives are left out.}

\wss{The objectives are important because they highlight that you are aware of 
limitations in your resources for verification and validation.  You can't do everything, 
so what are you going to prioritize?  As an example, if your system depends on an 
external library, you can explicitly state that you will assume that external library 
has already been verified by its implementation team.}

\subsection{Challenge Level and Extras}

\wss{State the challenge level (advanced, general, basic) for your project.
Your challenge level should exactly match what is included in your problem
statement.  This should be the challenge level agreed on between you and the
course instructor.  You can use a pull request to update your challenge level
(in TeamComposition.csv or Repos.csv) if your plan changes as a result of the
VnV planning exercise.}

\wss{Summarize the extras (if any) that were tackled by this project.  Extras
can include usability testing, code walkthroughs, user documentation, formal
proof, GenderMag personas, Design Thinking, etc.  Extras should have already
been approved by the course instructor as included in your problem statement.
You can use a pull request to update your extras (in TeamComposition.csv or
Repos.csv) if your plan changes as a result of the VnV planning exercise.}

\subsection{Relevant Documentation}

\wss{Reference relevant documentation.  This will definitely include your SRS
  and your other project documents (design documents, like MG, MIS, etc).  You
  can include these even before they are written, since by the time the project
  is done, they will be written.  You can create BibTeX entries for your
  documents and within those entries include a hyperlink to the documents.}

\citet{SRS}

\wss{Don't just list the other documents.  You should explain why they are relevant and 
how they relate to your VnV efforts.}

\section{Plan}

\wss{Introduce this section.  You can provide a roadmap of the sections to
  come.}

\subsection{Verification and Validation Team}

\wss{Your teammates.  Maybe your supervisor.
  You should do more than list names.  You should say what each person's role is
  for the project's verification.  A table is a good way to summarize this information.}

\subsection{SRS Verification}

\wss{List any approaches you intend to use for SRS verification.  This may
  include ad hoc feedback from reviewers, like your classmates (like your
  primary reviewer), or you may plan for something more rigorous/systematic.}

\wss{If you have a supervisor for the project, you shouldn't just say they will
read over the SRS.  You should explain your structured approach to the review.
Will you have a meeting?  What will you present?  What questions will you ask?
Will you give them instructions for a task-based inspection?  Will you use your
issue tracker?}

\wss{Maybe create an SRS checklist?}

\subsection{Design Verification}

\wss{Plans for design verification}

\wss{The review will include reviews by your classmates}

\wss{Create a checklists?}

\subsection{Verification and Validation Plan Verification}

\wss{The verification and validation plan is an artifact that should also be
verified.  Techniques for this include review and mutation testing.}

\wss{The review will include reviews by your classmates}

\wss{Create a checklists?}

\subsection{Implementation Verification}

\wss{You should at least point to the tests listed in this document and the unit
  testing plan.}

\wss{In this section you would also give any details of any plans for static
  verification of the implementation.  Potential techniques include code
  walkthroughs, code inspection, static analyzers, etc.}

\wss{The final class presentation in CAS 741 could be used as a code
walkthrough.  There is also a possibility of using the final presentation (in
CAS741) for a partial usability survey.}

\subsection{Automated Testing and Verification Tools}

\wss{What tools are you using for automated testing.  Likely a unit testing
  framework and maybe a profiling tool, like ValGrind.  Other possible tools
  include a static analyzer, make, continuous integration tools, test coverage
  tools, etc.  Explain your plans for summarizing code coverage metrics.
  Linters are another important class of tools.  For the programming language
  you select, you should look at the available linters.  There may also be tools
  that verify that coding standards have been respected, like flake9 for
  Python.}

\wss{If you have already done this in the development plan, you can point to
that document.}

\wss{The details of this section will likely evolve as you get closer to the
  implementation.}

\subsection{Software Validation}

\wss{If there is any external data that can be used for validation, you should
  point to it here.  If there are no plans for validation, you should state that
  here.}

\wss{You might want to use review sessions with the stakeholder to check that
the requirements document captures the right requirements.  Maybe task based
inspection?}

\wss{For those capstone teams with an external supervisor, the Rev 0 demo should 
be used as an opportunity to validate the requirements.  You should plan on 
demonstrating your project to your supervisor shortly after the scheduled Rev 0 demo.  
The feedback from your supervisor will be very useful for improving your project.}

\wss{For teams without an external supervisor, user testing can serve the same purpose 
as a Rev 0 demo for the supervisor.}

\wss{This section might reference back to the SRS verification section.}

\section{System Tests}

\wss{There should be text between all headings, even if it is just a roadmap of
the contents of the subsections.}

\subsection{Tests for Functional Requirements}

\wss{Subsets of the tests may be in related, so this section is divided into
  different areas.  If there are no identifiable subsets for the tests, this
  level of document structure can be removed.}

\wss{Include a blurb here to explain why the subsections below
  cover the requirements.  References to the SRS would be good here.}

\subsubsection{Area of Testing1}

\wss{It would be nice to have a blurb here to explain why the subsections below
  cover the requirements.  References to the SRS would be good here.  If a section
  covers tests for input constraints, you should reference the data constraints
  table in the SRS.}
		
\paragraph{Title for Test}

\begin{enumerate}

\item{test-id1\\}

Control: Manual versus Automatic
					
Initial State: 
					
Input: 
					
Output: \wss{The expected result for the given inputs.  Output is not how you
are going to return the results of the test.  The output is the expected
result.}

Test Case Derivation: \wss{Justify the expected value given in the Output field}
					
How test will be performed: 
					
\item{test-id2\\}

Control: Manual versus Automatic
					
Initial State: 
					
Input: 
					
Output: \wss{The expected result for the given inputs}

Test Case Derivation: \wss{Justify the expected value given in the Output field}

How test will be performed: 

\end{enumerate}

\subsubsection{Area of Testing2}

...

\subsection{Tests for Nonfunctional Requirements}

\wss{The nonfunctional requirements for accuracy will likely just reference the
  appropriate functional tests from above.  The test cases should mention
  reporting the relative error for these tests.  Not all projects will
  necessarily have nonfunctional requirements related to accuracy.}

\wss{For some nonfunctional tests, you won't be setting a target threshold for
passing the test, but rather describing the experiment you will do to measure
the quality for different inputs.  For instance, you could measure speed versus
the problem size.  The output of the test isn't pass/fail, but rather a summary
table or graph.}

\wss{Tests related to usability could include conducting a usability test and
  survey.  The survey will be in the Appendix.}

\wss{Static tests, review, inspections, and walkthroughs, will not follow the
format for the tests given below.}

\wss{If you introduce static tests in your plan, you need to provide details.
How will they be done?  In cases like code (or document) walkthroughs, who will
be involved? Be specific.}

\subsubsection{Area of Testing1}
		
\paragraph{Title for Test}

\begin{enumerate}

\item{test-id1\\}

Type: Functional, Dynamic, Manual, Static etc.
					
Initial State: 
					
Input/Condition: 
					
Output/Result: 
					
How test will be performed: 
					
\item{test-id2\\}

Type: Functional, Dynamic, Manual, Static etc.
					
Initial State: 
					
Input: 
					
Output: 
					
How test will be performed: 

\end{enumerate}

\subsubsection{Area of Testing2}

...

\subsection{Traceability Between Test Cases and Requirements}

\wss{Provide a table that shows which test cases are supporting which
  requirements.}

\section{Unit Test Description}

\wss{This section should not be filled in until after the MIS (detailed design
  document) has been completed.}

\wss{Reference your MIS (detailed design document) and explain your overall
philosophy for test case selection.}  

\wss{To save space and time, it may be an option to provide less detail in this section.  
For the unit tests you can potentially layout your testing strategy here.  That is, you 
can explain how tests will be selected for each module.  For instance, your test building 
approach could be test cases for each access program, including one test for normal behaviour 
and as many tests as needed for edge cases.  Rather than create the details of the input 
and output here, you could point to the unit testing code.  For this to work, you code 
needs to be well-documented, with meaningful names for all of the tests.}

\subsection{Unit Testing Scope}

\wss{What modules are outside of the scope.  If there are modules that are
  developed by someone else, then you would say here if you aren't planning on
  verifying them.  There may also be modules that are part of your software, but
  have a lower priority for verification than others.  If this is the case,
  explain your rationale for the ranking of module importance.}

\subsection{Tests for Functional Requirements}

\wss{Most of the verification will be through automated unit testing.  If
  appropriate specific modules can be verified by a non-testing based
  technique.  That can also be documented in this section.}

\subsubsection{Module 1}

\wss{Include a blurb here to explain why the subsections below cover the module.
  References to the MIS would be good.  You will want tests from a black box
  perspective and from a white box perspective.  Explain to the reader how the
  tests were selected.}

\begin{enumerate}

\item{test-id1\\}

Type: \wss{Functional, Dynamic, Manual, Automatic, Static etc. Most will
  be automatic}
					
Initial State: 
					
Input: 
					
Output: \wss{The expected result for the given inputs}

Test Case Derivation: \wss{Justify the expected value given in the Output field}

How test will be performed: 
					
\item{test-id2\\}

Type: \wss{Functional, Dynamic, Manual, Automatic, Static etc. Most will
  be automatic}
					
Initial State: 
					
Input: 
					
Output: \wss{The expected result for the given inputs}

Test Case Derivation: \wss{Justify the expected value given in the Output field}

How test will be performed: 

\item{...\\}
    
\end{enumerate}

\subsubsection{Module 2}

...

\subsection{Tests for Nonfunctional Requirements}

The following will outline system tests for the nonfunctional requirements highlighted in the SRS, directly verifying each Fit Criterion. Note that there are tests that cover multiple nonfunctional requirements. At the same time, some nonfunctional requirements are not present as they are already accounted for in the functional requirements test plan. 

\subsubsection{Look and Feel}
\begin{enumerate}[label=\bfseries LF-\arabic*:, wide=0pt]
  \item \label{test-LF1} \textbf{MES Branding Conformance}\\[2mm]
    {\bf Type:} Manual, Static\\
    {\bf Initial State:} UI for basic build is finalized.\\
    {\bf Input/Condition:} MES representative given access to app UI.\\
    {\bf Output/Result:} Approval from MES representative.\\
    {\bf Covers:} AR-1\\
    {\bf How test will be performed:} A meeting will be held with the representative with a clear showing of the app UI, where they will assess design decisions (branding, colour palette) against MES branding criteria.
  
  \item \label{test-LF2} \textbf{Design Reception Survey}\\[2mm]
    {\bf Type:} Manual, Dynamic\\
    {\bf Initial State:} Basic production build and survey ready.\\
    {\bf Input/Condition:} App used in at least one MES event, and survey sent out to all attendees.\\
    {\bf Output/Result:} Survey filled out to illustrate user reception on design.\\
    {\bf Covers:} AR-2, SR-1\\
    {\bf How test will be performed:} The survey results will be compiled and analyzed to determine user reception of the design, and whether it meets usability expectations.

    \item \label{test-LF3} \textbf{Admin Analytics Acceptance}\\[2mm]
    {\bf Type:} Manual, Dynamic\\
    {\bf Initial State:} Admin dashboard operational and contains realistic analytics and charts.\\
    {\bf Input/Condition:} MES representative given access to dashboard preview.\\
    {\bf Output/Result:} Approval from MES representative.\\
    {\bf Covers:} AR-3, SR-2, FT-5\\
    {\bf How test will be performed:} A meeting will be held with the representative to demonstrate the dashboard analytics. They will give approval on the types of metrics visualized as well as their layout.
\end{enumerate}

\subsubsection{Usability and Humanity}
\begin{enumerate}[label=\bfseries UH-\arabic*:, wide=0pt]
  \item \label{test-UH1} \textbf{Intuitive Feedback and Registration}\\[2mm]
    {\bf Type:} Manual, Dynamic\\
    {\bf Initial State:} Functional live event forms.\\
    {\bf Input/Condition:} App used in at least one MES event, attendees are provided access to app.\\
    {\bf Output/Result:} Successful submission of pertinent forms with no instruction, and highly rating the experience.\\
    {\bf Covers:} ER-1, LR-1\\
    {\bf How test will be performed:} Observe successful registration and form completion by minimum 30 attendees, with an average of 80\% rating ease of use a 4+ out of 5.(ER-1).
  
  \item \label{test-UH2} \textbf{Seamless Sign-In / Sign-Up}\\[2mm]
    {\bf Type:} Manual, Dynamic\\
    {\bf Initial State:} Prior to set up of new user accounts.\\
    {\bf Input/Condition:} Users perform sign-up and sign-in with no instruction.\\
    {\bf Output/Result:} 10 users complete registration and sign in to account.\\
    {\bf Covers:} ER-2\\
    {\bf How test will be performed:} A sample of 10 users will be taken and observed for their ability to complete registration and login within 2 minutes (ER-2).

  \item \label{test-UH3} \textbf{Admin Dashboard and Form Building Usability}\\[2mm]
    {\bf Type:} Manual, Dynamic\\
    {\bf Initial State:} Access to admin account with no prior training.\\
    {\bf Input/Condition:} Admin creates a basic event form, and interprets analytics.\\
    {\bf Output/Result:} 10 users successfully create event forms, and illustrate correct understanding of analytics.\\
    {\bf Covers:} SR-2, ER-3, PI-1, PI-2, LR-2\\
    {\bf How test will be performed:} Admins are instructed to create a basic event form, and describe their understanding of the analytics dashboard. A minimum of 10 users will be assessed for successful form creation and understanding of analytics (ER-3/SR-2).

  \item \label{test-UH4} \textbf{Undesrtandable Language and Clear Prompts}\\[2mm]
    {\bf Type:} Manual, Dynamic\\
    {\bf Initial State:} All prompts implemented.\\
    {\bf Input/Condition:} Used in at least one MES event where users interact with app prompts and evaluate clarity of wording.\\
    {\bf Output/Result:} 80\% of users complete survey and rate language as accetptable.\\
    {\bf Covers:} UR-1, UR-2, UR-4, CL-1\\
    {\bf How test will be performed:} After completing the registration process for the MES event, users will be prompted to complete a feedback survey, where they rate language clarity out of 5.

  \item \label{test-UH5} \textbf{Polite and Constructive Error Handling}\\[2mm]
    {\bf Type:} Manual, Dynamic\\
    {\bf Initial State:} Platform triggers common errors.\\
    {\bf Input/Condition:} User experiences error message and evaluate usefulness.\\
    {\bf Output/Result:} 80\% of users rate the experience as satisfying.\\
    {\bf Covers:} UR-1, UR-2\\
    {\bf How test will be performed:} Users will aim to fill out the forms for at least one MES event, and interact with error messages to direct the completion of the form. Afterwards, they will rate the usefulness of the error messages in a survey out of 5.
  
  \item \label{test-UH6} \textbf{Accessability Conformance}\\[2mm]
    {\bf Type:} Automatic, Dynamic\\
    {\bf Initial State:} Fully configured UI themes.\\
    {\bf Input/Condition:} Web tool used to assess UI.\\
    {\bf Output/Result:} Design conforms to WCAG 2.0 and AODA requirements.\\
    {\bf Covers:} AC-1\\
    {\bf How test will be performed:} Automated web tools will be used to assess conformance to accessability standards (WebAIM). 
\end{enumerate}

\subsection{Performance}
\begin{enumerate}[label=\bfseries PF-\arabic*:, wide=0pt]
  \item \label{test-PF1} \textbf{Form Latency}\\[2mm]
    {\bf Type:} Automatic, Dynamic\\
    {\bf Initial State:} Platform open on variety of devices (iOS, Android, Windows).\\
    {\bf Input/Condition:} Loading event forms.\\
    {\bf Output/Result:} Measured Time to First Byte (TTFB) will be less than 2 seconds.\\
    {\bf Covers:} SL-1, IR-1, IR-2, PD-1, PD-2, AD-1, AD-2\\
    {\bf How test will be performed:} App will be opened on distinct devices, and the according browser developer tools and APM's will be used to measure whether form loading latency meets performance requirements.

  \item \label{test-PF2} \textbf{Data and Analytics Latency}\\[2mm]
    {\bf Type:} Automatic, Dynamic\\
    {\bf Initial State:} Admin dashboard open on variety of devices (iOS, Android, Windows).\\
    {\bf Input/Condition:} Loading data analytics and performing filtering queries.\\
    {\bf Output/Result:} Measured Time to First Byte (TTFB) using browser developer tools and APM's.\\
    {\bf Covers:} SL-2, IR-1, IR-2, PD-1, PD-2\\
    {\bf How test will be performed:} App will be opened on distinct devices, and the according browser developer tools and APM's will be used to measure whether analytics latency meets performance requirements.

  \item \label{test-PF4} \textbf{Wide User and Data Capacity}\\[2mm]
    {\bf Type:} Automatic, Dynamic\\
    {\bf Initial State:} Fully configured user access.\\
    {\bf Input/Condition:} Stress tool used to simulate high traffic.\\
    {\bf Output/Result:} Assess tool results to determine platform performance and bottlenecks.\\
    {\bf Covers:} CR-1, CR-2\\
    {\bf How test will be performed:} Load and stress tools (JMeter, DevTools) will be used to determine performance capacity of the app and areas of improvement to increase limitations.  

{Untested Requirements}
\begin{itemize}
    \item \textbf{Safety Critical (SC-1):} Verified by \textbf{Test-FR-AP-2}, which compare dashboard values to backend analytics and exported reports.

    \item \textbf{Precision or Accuracy (PA-1):}Verified by \textbf{Test-FR-AP-2} and \textbf{Test-FR-AP-4}, which compare dashboard values to backend analytics and exported reports.
  
    \item \textbf{Robustness or Fault Tolerance (PA-1, PA-2, PA-3, PA-4):} Verified by \textbf{Test-FR-SI-1}, which tests data synchronization and ensures all updates are saved and automatically viewable.
    
    \item \textbf{Scalability or Extensibility (SE-2):} Verified through code review. This is not directly measureable through system testing.

    \item \textbf{Longevity (LG-2):} \textbf{Test-FR-AP-4} confirms the report/export function.
\end{itemize}

\end{enumerate}

\subsubsection{Operational and Evironmental}
\begin{enumerate}[label=\bfseries OE-\arabic*:, wide=0pt]
  \item \label{test-OE1} \textbf{Timely Prototype Completion}\\[2mm]
    {\bf Type:} Manual, Static\\
    {\bf Initial State:} Completion of basic prototype.\\
    {\bf Input/Condition:} Give MES representative access to prototype.\\
    {\bf Output/Result:} Receive approval.\\
    {\bf Covers:} RR-1\\
    {\bf How test will be performed:} The MES representative will be given access to the basic prototype, and a meeting will be held to receive their approval on whether it meets expectations for the initial release.
\end{enumerate}

\subsubsection{Maintainability and Support}
\begin{enumerate}[label=\bfseries MS-\arabic*:, wide=0pt]
  \item \label{test-MS1} \textbf{Updated Product Documentation}\\[2mm]
    {\bf Type:} Manual, Static\\
    {\bf Initial State:} Latest release of product available.\\
    {\bf Input/Condition:} Developer references product documentation.\\
    {\bf Output/Result:} Documentation will be followed to add new features or correct errors.\\
    {\bf Covers:} MT-1\\
    {\bf How test will be performed:} Developers will follow documentation to identify the changes made with each release and direct future modification processes.

  \item \label{test-MS2} \textbf{FAQ Coverage}\\[2mm]
    {\bf Type:} Manual, Static\\
    {\bf Initial State:} Published FAQ section.\\
    {\bf Input/Condition:} Crosscheck FAQ postings with stakeholder/representative questions.\\
    {\bf Output/Result:} Listing of ideal postings for FAQ section.\\
    {\bf Covers:} SU-1\\
    {\bf How test will be performed:} Compile stakeholder feedback and survey results and compare to FAQ section to validate inclusion of pertinent topics.
\end{enumerate}

\subsubsection{Security}
\begin{enumerate}[label=\bfseries ST-\arabic*:, wide=0pt]

  \item \label{test-ST1} \textbf{Secure Email Authentication}\\[2mm]
    {\bf Type:} Manual, Dynamic\\
    {\bf Initial State:} Access to app with authentication configured to McMaster emails.\\
    {\bf Input/Condition:} Attempt login with non-McMaster and McMaster emails.\\
    {\bf Output/Result:} Only authorized users pass. Any invalid attempts are logged for admin to view\\
    {\bf Covers:} AC-1\\
    {\bf How test will be performed:} App login will be tested with valid and invalid email addresses. Only the valid should be able to access the app. The invalid attempt should be logged in history.\\

  \item \label{test-ST2} \textbf{HTTPS-Limited Communication}\\[2mm]
    {\bf Type:} Manual, Dynamic\\
    {\bf Initial State:} TLS configured domain.\\
    {\bf Input/Condition:} Attempt HTTP connections and inspect traffic.\\
    {\bf Output/Result:} All HTTP traffic redirects to HTTPS.\\
    {\bf Covers:} IG-2\\
    {\bf How test will be performed:} Different connections from different devices will be attempted, followed by inspections of the browser logs to verify correct use of HTTPS.\\

  \item \label{test-ST3} \textbf{Personal Data Deletion}\\[2mm]
    {\bf Type:} Manual, Dynamic\\
    {\bf Initial State:} User account already made with personal data stored in database.\\
    {\bf Input/Condition:} Submit a request to delete data through the app.\\
    {\bf Output/Result:} All personal data removed fairly in accordance with policy.\\
    {\bf Covers:} PV-1\\
    {\bf How test will be performed:} User will create an account, and request deletion after confirming its data was stored in the database. After alleged deletion, query the database to verify complete removal of data.\\

  \item \label{test-ST4} \textbf{Secure Data Encryption (AES-256)}\\[2mm]
    {\bf Type:} Manual, Static\\
    {\bf Initial State:} No account created yet.\\
    {\bf Input/Condition:} Send account information to database.\\
    {\bf Output/Result:} Data entry is fully encrypted with no plaintext values.\\
    {\bf Covers:} PV-2, PV-3\\
    {\bf How test will be performed:} Test account will be created, followed by an inspection of the data entry to ensure competent encryption.\\

  \item \label{test-ST5} \textbf{Detailed Admin Audit Log}\\[2mm]
    {\bf Type:} Manual, Dynamic\\
    {\bf Initial State:} Audit logging fully enabled.\\
    {\bf Input/Condition:} Perform different admin actions (create, update, delete, export).\\
    {\bf Output/Result:} Each action recorded with actor, timestamp, and action type.\\
    {\bf Covers:} AU-1\\
    {\bf How test will be performed:} Replay test scenarios and confirm completeness and accuracy of logs.\\

{Untested Requirements}
\begin{itemize}
    \item \textbf{Access (AC-2):} Verified by \textbf{Test-FR-SI-2}, which ensures restriction of data based on permissions.

    \item \textbf{Integrity (IG-1):}Verified by \textbf{Test-FR-SI-1}, which covers data synchronization and accuracy of inputs.
\end{itemize}

\end{enumerate}

\subsubsection{Compliance}
\begin{enumerate}[label=\bfseries ST-\arabic*:, wide=0pt]
  \item \label{test-ST1} \textbf{Updated Product Documentation}\\[2mm]
    {\bf Type:} Manual, Static\\
    {\bf Initial State:} Latest release of product available.\\
    {\bf Input/Condition:} Developer references product documentation.\\
    {\bf Output/Result:} Documentation will be followed to add new features or correct errors.\\
    {\bf Covers:} MT-1\\
    {\bf How test will be performed:} Developers will follow documentation to identify the changes made with each release and direct future modification processes.
\end{enumerate}

...

\subsection{Traceability Between Test Cases and Modules}

Traceability between the unit test cases and the modules they are intended to verify is already illustrated in the \textbf{Covers} section for each NFR.

\bibliographystyle{plainnat}

\bibliography{../../refs/References}

\newpage

\section{Appendix}

This is where you can place additional information.

\subsection{Symbolic Parameters}

The definition of the test cases will call for SYMBOLIC\_CONSTANTS.
Their values are defined in this section for easy maintenance.

\subsection{Usability Survey Questions?}

\wss{This is a section that would be appropriate for some projects.}

\newpage{}
\section*{Appendix --- Reflection}

\wss{This section is not required for CAS 741}

The information in this section will be used to evaluate the team members on the
graduate attribute of Lifelong Learning.

The purpose of reflection questions is to give you a chance to assess your own
learning and that of your group as a whole, and to find ways to improve in the
future. Reflection is an important part of the learning process.  Reflection is
also an essential component of a successful software development process.  

Reflections are most interesting and useful when they're honest, even if the
stories they tell are imperfect. You will be marked based on your depth of
thought and analysis, and not based on the content of the reflections
themselves. Thus, for full marks we encourage you to answer openly and honestly
and to avoid simply writing ``what you think the evaluator wants to hear.''

Please answer the following questions.  Some questions can be answered on the
team level, but where appropriate, each team member should write their own
response:


\begin{enumerate}
  \item What went well while writing this deliverable? 
  \item What pain points did you experience during this deliverable, and how
    did you resolve them?
  \item What knowledge and skills will the team collectively need to acquire to
  successfully complete the verification and validation of your project?
  Examples of possible knowledge and skills include dynamic testing knowledge,
  static testing knowledge, specific tool usage, Valgrind etc.  You should look to
  identify at least one item for each team member.
  \item For each of the knowledge areas and skills identified in the previous
  question, what are at least two approaches to acquiring the knowledge or
  mastering the skill?  Of the identified approaches, which will each team
  member pursue, and why did they make this choice?
\end{enumerate}

\end{document}