\documentclass{article}

\usepackage{booktabs}
\usepackage{tabularx}
\usepackage{comment}

\title{Development Plan\\\progname}

\author{\authname}

\date{}

%% Comments

\usepackage{color}

\newif\ifcomments\commentstrue %displays comments
%\newif\ifcomments\commentsfalse %so that comments do not display

\ifcomments
\newcommand{\authornote}[3]{\textcolor{#1}{[#3 ---#2]}}
\newcommand{\todo}[1]{\textcolor{red}{[TODO: #1]}}
\else
\newcommand{\authornote}[3]{}
\newcommand{\todo}[1]{}
\fi

\newcommand{\wss}[1]{\authornote{magenta}{SS}{#1}} 
\newcommand{\plt}[1]{\authornote{cyan}{TPLT}{#1}} %For explanation of the template
\newcommand{\an}[1]{\authornote{cyan}{Author}{#1}}

%% Common Parts

\newcommand{\progname}{Software Engineering} % PUT YOUR PROGRAM NAME HERE
\newcommand{\teamname}{EvENGage}
\newcommand{\authname}{Team 4, \teamname
\\ Virochaan Ravichandran Gowri
\\ Omar Al-Asfar
\\ Rayyan Suhail
\\ Ibrahim Quraishi
\\ Mohammad Mahdi Mahboob} % AUTHOR NAMES

\newcommand{\prjdesc}{MES Event Management Registration, Administration, and Survey Analytics}

\usepackage{hyperref}
    \hypersetup{colorlinks=true, linkcolor=blue, citecolor=blue, filecolor=blue,
                urlcolor=blue, unicode=false}
    \urlstyle{same}



\begin{document}

\maketitle
\begin{table}[hp]
\caption{Revision History} \label{TblRevisionHistory}
\begin{tabularx}{\textwidth}{llX}
\toprule
\textbf{Date} & \textbf{Developer(s)} & \textbf{Change}\\
\midrule
September 21, 2025 & Ibrahim, Mohammad, Omar & Revision 0 \\
\bottomrule
\end{tabularx}
\end{table}

\newpage{}

This document will outline the development plan for \textbf{\teamname}:
handling of confidential information; IP protection and copyright licensing;
team roles, collaboration, and organization guidelines; project development
guidelines; project workflow and version control guidelines; and expected
programming tools, technologies, and standards.
\\
This project will be conducted in collaboration with two other Capstone teams
to create one large unified product, and as such, the contents in these
sections are subject to change based on updated requirements from the project
supervisors or design decisions made in conjunction with the collaborating
teams.

\begin{comment}
\wss{Additional information on the development plan can be found in the
\href{https://gitlab.cas.mcmaster.ca/courses/capstone/-/blob/main/Lectures/L02b_POCAndDevPlan/POCAndDevPlan.pdf?ref_type=heads}
{lecture slides}.}
\end{comment}

\section{Confidential Information?}

\begin{flushleft}
The project will deal with information confidential to the McMaster Engineering Society (MES) and the Canadian Federation of Engineering Students (CFES). 
This information includes program generated analytics and financial reports for events hosted by the MES through the proposed system, and results from surveys conducted by the CFES.
\end{flushleft}

\section{IP to Protect}

\begin{comment}
\wss{State whether there is IP to protect.  If there is, point to the agreement.
All students who are working on a project that requires an IP agreement are also
required to sign the ``Intellectual Property Guide Acknowledgement.''}
\end{comment}

There is currently no IP which requires protection, and the project is planned
to be open-source.

\section{Copyright License}

\begin{comment}
\wss{What copyright license is your team adopting.  Point to the license in your
repo.}
\end{comment}

This project will use the GNU General Public License version 3 (GPLv3). The
license can be found \href{https://github.com/VirochaanRG/MES-Event-Management-System/blob/main/LICENSE}{here}.

\section{Team Meeting Plan}

\begin{flushleft}
The members of the team are expected to meet at least once weekly virtually through Microsoft Teams. 
The purpose of these meetings is to provide updates on current work, plan accordingly for submission of deliverables, discuss next steps, and distribute tasks among team members. 
The structure of this meeting will be as follows.
  \begin{enumerate}
    \item Each team member will provide a 5–6-minute update on the work completed within the last week. After each member has completed their update, the other team members are free to provide feedback.
    \item The meeting leader will go over next steps on current deliverables, as well as remind team members of upcoming deliverables
    \item If any new work is to be assigned, the meeting leader will split the work into a set of tasks. Team members will then select which tasks they are comfortable with taking
    \item The remaining meeting time will be spent on preparation for the upcoming sync meeting with the supervisors
  \end{enumerate}
Additionally, a weekly sync meeting will be scheduled with the industry advisor and team supervisor to ensure that the direction of the project remains aligned with the needs of the supervisors. 
The meeting will ensure the supervisors are kept up to date with the state of the project and allow the supervisors to provide feedback or request changes. 
If there are no significant updates within the week before the meeting (i.e. during exam season), the meeting will be cancelled with at least 24 hours notice for the supervisors.
\end{flushleft}

\section{Team Communication Plan}

\wss{Issues on GitHub should be part of your communication plan.}

\section{Team Member Roles}

\begin{flushleft}
The following administrative roles will be assigned to team members:
\begin{itemize}
\item Team Liason: Main contact point between the team members and the supervisors. 
This member is responsible for sending emails for communication between the team and supervisors as well as scheduling and leading the weekly sync meetings.
\item Inter-Team Liason: Responsible for communication between capstone teams. 
Since this project is a sub-component of a larger system for the MES with multiple teams working together, it is important to have a representative for the team who will be responsible for communication between teams to ensure consistency and compatibility between design decisions for the project.
\item Meeting Leader: Responsible for facilitating the weekly meetings between team members. 
Ensures that team meetings happen consistently and that team members are up to date with deliverables.
\item Note Taker: Responsible for taking notes summarizing key points from every meeting, including team meetings and supervising meetings. 
This member is also responsible for creating GitHub Issues for every meeting and providing a summary of attendance and key takeaways.
\item Reviewer: Responsible for final review of any main deliverables before final submission to ensure they align with the course and supervisors’ expectations.
\end{itemize}
\end{flushleft}

\section{Workflow Plan}

\begin{itemize}
	\item How will you be using git, including branches, pull request, etc.?
	\item How will you be managing issues, including template issues, issue
	classification, etc.?
  \item Use of CI/CD
\end{itemize}

\section{Project Decomposition and Scheduling}

\begin{itemize}
  \item How will you be using GitHub projects?
  \item Include a link to your GitHub project
\end{itemize}

\wss{How will the project be scheduled?  This is the big picture schedule, not
details. You will need to reproduce information that is in the course outline
for deadlines.}

\section{Proof of Concept Demonstration Plan}

\begin{comment}
What is the main risk, or risks, for the success of your project?  What will you
demonstrate during your proof of concept demonstration to convince yourself that
you will be able to overcome this risk?
\end{comment}

The main challenges we foresee for this project are creating the form builder
and integrating user-side and admin-side interfaces for the application. The
form-builder needs to be modular and track all data against a user to perform
analytics in order to minimize redundancy and improve user experience. The
interface integration is also challenging because they demand an entirely
different set of features for the client, yet they must interact with the same
underlying system.
\\
To get a working proof-of-concept, we expect to demonstrate a simplified interface
with at least singular interactions from either endpoints which can communicate
correctly with the underlying database. If the database is correctly and
efficiently track all data, then we can expect that the final product can be
developed with confidence.

\section{Expected Technology}

\begin{flushleft}
At the request of the supervisor, the following tools and technologies must be used for development of the project to ensure compatibility and seamless integration between capstone teams.
\begin{itemize}
    \item JavaScript, and HTML/CSS shall be used in development of the frontend for both the web-based admin portal, and the user mobile application. 
    React and Next.js will be used as the primary framework for the admin portal and React Native will be used to develop the mobile application.
    \item PostgreSQL shall be used as the primary database for storage of any system and user data.
\end{itemize}

Additionally, the following programming languages, frameworks and libraries are expected to be used during development and may change at the discretion of the team.
\begin{itemize}
    \item JavaScript along with Node.js may be used in development of the backend server for both the web portal and mobile application. 
    Additionally, Node.js modules such as node-postgres may be used to communicate with the database.
    \item A styling library/framework such as Tailwind or Bootstrap may be used to simplify UI styling
    \item A REST API will be used for communication between the frontend applications and the backend server. 
    The default Fetch API provided within React and React Native may be used on the frontend, along with Node.js modules such as Express to handle requests on the backend server
    \item Jest will be used as the primary unit testing framework for testing both the frontend and backend code with JavaScript, including unit testing and code coverage. 
    Additionally, the React Testing Library may be used to test frontend components.
    \item A linter such as Prettier will be used to ensure the source code adheres to a specific standard.
\end{itemize}

GitHub will be used as the project repository for documentation, project planning and administration, and source code management.
\begin{itemize}
    \item A GitHub Issue will be created for each attended lecture, tutorial, and team meeting outlining attendance, and key takeaways. 
    \item A GitHub Issue will be created for each course deliverable, such as documentation, and code submissions
    \item GitHub Projects will be used to host a Kanban Board of all GitHub Issues and their status
    \item GitHub Actions will be used as the main CI platform. CI will be used to maintain code styling requirements and perform regression tests on pull requests before they can be merged. Additionally, GitHub actions may be used build and deploy source code into a staging environment.
    \item Git will be used as the source code management platform for making code changes. The specific development environment in which these changes are made, such as the IDE, are subject to the preferences of the team members.
\end{itemize}
\end{flushleft}

\section{Coding Standard}

\wss{What coding standard will you adopt?}

\newpage{}

\section*{Appendix --- Reflection}

\wss{Not required for CAS 741}

The purpose of reflection questions is to give you a chance to assess your own
learning and that of your group as a whole, and to find ways to improve in the
future. Reflection is an important part of the learning process.  Reflection is
also an essential component of a successful software development process.  

Reflections are most interesting and useful when they're honest, even if the
stories they tell are imperfect. You will be marked based on your depth of
thought and analysis, and not based on the content of the reflections
themselves. Thus, for full marks we encourage you to answer openly and honestly
and to avoid simply writing ``what you think the evaluator wants to hear.''

Please answer the following questions.  Some questions can be answered on the
team level, but where appropriate, each team member should write their own
response:


\begin{enumerate}
    \item Why is it important to create a development plan prior to starting the
    project?
    \item In your opinion, what are the advantages and disadvantages of using
    CI/CD?
    \item What disagreements did your group have in this deliverable, if any,
    and how did you resolve them?
\end{enumerate}

\newpage{}

\section*{Appendix --- Team Charter}

\wss{borrows from
\href{https://engineering.up.edu/industry_partnerships/files/team-charter.pdf}
{University of Portland Team Charter}}

\subsection*{External Goals}

\wss{What are your team's external goals for this project? These are not the
goals related to the functionality or quality fo the project.  These are the
goals on what the team wishes to achieve with the project.  Potential goals are
to win a prize at the Capstone EXPO, or to have something to talk about in
interviews, or to get an A+, etc.}

\subsection*{Attendance}

\subsubsection*{Expectations}

\wss{What are your team's expectations regarding meeting attendance (being on
time, leaving early, missing meetings, etc.)?}

\subsubsection*{Acceptable Excuse}

\wss{What constitutes an acceptable excuse for missing a meeting or a deadline?
What types of excuses will not be considered acceptable?}

\subsubsection*{In Case of Emergency}

\wss{What process will team members follow if they have an emergency and cannot
attend a team meeting or complete their individual work promised for a team
deliverable?}

\subsection*{Accountability and Teamwork}

\subsubsection*{Quality} 

\wss{What are your team's expectations regarding the quality
of team members' preparation for team meetings and the quality of the
deliverables that members bring to the team?}

\subsubsection*{Attitude}

\wss{What are your team's expectations regarding team members' ideas,
interactions with the team, cooperation, attitudes, and anything else regarding
team member contributions?  Do you want to introduce a code of conduct?  Do you
want a conflict resolution plan?  Can adopt existing codes of conduct.}

\subsubsection*{Stay on Track}

\wss{What methods will be used to keep the team on track? How will your team
ensure that members contribute as expected to the team and that the team
performs as expected? How will your team reward members who do well and manage
members whose performance is below expectations?  What are the consequences for
someone not contributing their fair share?}

\wss{You may wish to use the project management metrics collected for the TA and
instructor for this.}

\wss{You can set target metrics for attendance, commits, etc.  What are the
consequences if someone doesn't hit their targets?  Do they need to bring the
coffee to the next team meeting?  Does the team need to make an appointment with
their TA, or the instructor?  Are there incentives for reaching targets early?}

\subsubsection*{Team Building}

\wss{How will you build team cohesion (fun time, group rituals, etc.)? }

\subsubsection*{Decision Making} 

\wss{How will you make decisions in your group? Consensus?  Vote? How will you
handle disagreements? }

\end{document}