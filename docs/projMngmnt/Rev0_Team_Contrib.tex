\documentclass{article}

\usepackage{float}
\restylefloat{table}

\usepackage{booktabs}

\title{Team Contributions: Rev 0\\\progname}

\author{\authname}

\date{}

\input{../Comments}
%% Common Parts

\newcommand{\progname}{Software Engineering} % PUT YOUR PROGRAM NAME HERE
\newcommand{\authname}{Team 4, EventHub
\\ Virochaan Ravichandran Gowri
\\ Omar Al-Asfar
\\ Rayyan Suhail
\\ Ibrahim Quraishi
\\ Mohammad Mahdi Mahboob} % AUTHOR NAMES                  

\usepackage{hyperref}
    \hypersetup{colorlinks=true, linkcolor=blue, citecolor=blue, filecolor=blue,
                urlcolor=blue, unicode=false}
    \urlstyle{same}
                                


\begin{document}

\maketitle

This document summarizes the contributions of each team member for the Rev 0
Demo.  The time period of interest is the time between the PoC demo and the Rev
0 demo; the contributions prior to the PoC are NOT included.

\section{Demo Plans}



\section{Team Meeting Attendance}


\begin{table}[H]
\centering
\begin{tabular}{ll}
\toprule
\textbf{Student} & \textbf{Meetings}\\
\midrule
Total & 8\\
Ibrahim Quraishi & 8\\
Virochaan Ravichandran Gowri & 8\\
Mohammad Mahdi Mahboob& 8\\
Rayyan Suhail& 8\\
Omar Al-Asfar& 8\\
\bottomrule
\end{tabular}
\end{table}



\section{Supervisor/Stakeholder Meeting Attendance}

\noindent \textbf{Supervisor's Name: } Luke Schuurman

\begin{table}[H]
\centering
\begin{tabular}{ll}
\toprule
\textbf{Student} & \textbf{Meetings}\\
\midrule
Total & 4\\
Ibrahim Quraishi & 4\\
Virochaan Ravichandran Gowri & 4\\
Mohammad Mahdi Mahboob& 4\\
Rayyan Suhail& 4\\
Omar Al-Asfar& 4\\
\bottomrule
\end{tabular}
\end{table}


\section{Lecture Attendance}

\begin{table}[H]
\centering
\begin{tabular}{ll}
\toprule
\textbf{Student} & \textbf{Lectures}\\
\midrule
Total & 9\\
Ibrahim Quraishi & 7\\
Virochaan Ravichandran Gowri & 3\\
Mohammad Mahdi Mahboob & 3\\
Rayyan Suhail & 3\\
Omar Al-Asfar & 3\\
\bottomrule
\end{tabular}
\end{table}

\wss{If needed, an explanation for the lecture attendance can be provided here.}

\section{TA Document Discussion Attendance}


\noindent \textbf{TA's Name: } Tiago de Moraes Machado

\begin{table}[H]
\centering
\begin{tabular}{ll}
\toprule
\textbf{Student} & \textbf{Lectures}\\
\midrule
Total & 3\\
Ibrahim Quraishi & 3\\
Virochaan Ravichandran Gowri & 3\\
Mohammad Mahdi Mahboob & 3\\
Rayyan Suhail & 3\\
Omar Al-Asfar & 3\\
\bottomrule
\end{tabular}
\end{table}

The final TA meeting for the Design Documentation was cancelled due to a snow day.

\section{Commits}

\begin{table}[H]
\centering
\begin{tabular}{lll}
\toprule
\textbf{Student} & \textbf{Commits} & \textbf{Percent}\\
\midrule
Total & 413 & 100\% \\
Ibrahim Quraishi & 81 & 19.6\% \\
Virochaan Ravichandran Gowri & 143 & 34.6\% \\
Rayyan Suhail & 53 & 12.8\% \\
Mohammad Mahdi Mahboob & 72 & 17.4\% \\
Omar Al-Asfar & 64 & 15.5\% \\
\bottomrule
\end{tabular}
\end{table}

While there are differences in contribution percentages, there was in fact an improvement compared to the POC Team Contribution report. The remaining differences can be attributed to the same reasoning as before, with members commiting at different frequencies. For instance, Omar has the 2nd lowest commit count, but contributed the 2nd highest in terms of lines of code. Mohammad's contributions followed a similar pattern. In addition, Virochaan, who had the highest commit percentage and line count, was responsible for integrating provided templates and handling merging/pull requests, resulting in a higher number of commits.

\section{Issue Tracker}

\wss{For each team member how many issues have they authored (including open and
closed issues (O+C)) and how many have they been assigned (only counting closed
issues (C only)) over the time period of interest.}

\begin{table}[H]
\centering
\begin{tabular}{lll}
\toprule
\textbf{Student} & \textbf{Authored (O+C)} & \textbf{Assigned (C only)}\\
\midrule
Name 1 & Num & Num \\
Name 2 & Num & Num \\
Name 3 & Num & Num \\
Name 4 & Num & Num \\
Name 5 & Num & Num \\
\bottomrule
\end{tabular}
\end{table}

\wss{If needed, an explanation for the counts can be provided here.}

\section{CICD}

CI/CI is still intended to be implemented in a later phase of the project. It's primary application remains to be to support regression testing, with the pipeline accomodating verified unit tests to ensure functionality over time. It will also be used to ensure compliance with Airbnb JavaScript and JSX guidelines. Overall, the team has made progress in setting up the necessary infrastructure and is actively working towards the deployment of the CI/CD pipeline.

\section{Team Charter Trigger Items}

\wss{Provide a summary of the quantified triggers identified in the team's
charter.}

\wss{Provide a list of any violations of the triggers.  If the team wishes, the
violations can be summarized on aggregate, instead of naming specific team
members.}

\wss{Provide a plan to address the violations.  This could include revising the
triggers, if they are found to be too weak, strong or ambiguous.}

\section{Additional Productivity Metrics}

The additional productivity metrics we considered previously were pull requests and task completion time. The latter was deemed an ineffective measure due to the varying complexity of tasks, which made it difficult to draw meaningful conclusions. The pull request contributions are as follows: 

\begin{table}[H]
\centering
\begin{tabular}{lll}
\toprule
\textbf{Student} & \textbf{Pull Requests} & \textbf{Percent}\\
\midrule
Total & 71 & 100\% \\
Ibrahim Quraishi & 11 & 15.5\% \\
Virochaan Ravichandran Gowri & 17 & 24.0\% \\
Rayyan Suhail & 17 & 24.0\% \\
Mohammad Mahdi Mahboob & 13 & 18.3\% \\
Omar Al-Asfar & 13 & 18.3\% \\
\bottomrule
\end{tabular}
\end{table}

\noindent We can see that this metric demonstrates a more balanced contribution percentage compared to commits and issues. Pull requests are a more broad action, generally created per major task, and consisting of multiple commits and issues. This makes it a more reliable indicator of team contributions. There are still discrepancies due to varying task complexities, and members creating several pull requests per task, but it is likely more resistant to inflation than the other metrics. 
\\\\
\noindent In the future, it may be beneficial to separately track coding and documentation pull requests to more fairly assess contributions. Coding tasks typically require more effort and time, so they should be valued more heavily than documentation tasks.

\end{document}