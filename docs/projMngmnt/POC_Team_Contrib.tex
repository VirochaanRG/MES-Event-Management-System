\documentclass{article}

\usepackage{float}
\restylefloat{table}

\usepackage{booktabs}

\title{Team Contributions: POC\\\progname}

\author{\authname}

\date{}

\input{../Comments}
%% Common Parts

\newcommand{\progname}{Software Engineering} % PUT YOUR PROGRAM NAME HERE
\newcommand{\authname}{Team 4, EventHub
\\ Virochaan Ravichandran Gowri
\\ Omar Al-Asfar
\\ Rayyan Suhail
\\ Ibrahim Quraishi
\\ Mohammad Mahdi Mahboob} % AUTHOR NAMES                  

\usepackage{hyperref}
    \hypersetup{colorlinks=true, linkcolor=blue, citecolor=blue, filecolor=blue,
                urlcolor=blue, unicode=false}
    \urlstyle{same}
                                


\begin{document}

\maketitle

This document summarizes the contributions of each team member up to the POC
Demo.  The time period of interest is the time between the beginning of the term
and the POC demo.

\section{Demo Plans}

For the POC demo, we will be spending a week completing the design documentation, and the remaining week before the deadline working on a codebase for the proof of concept. The goal for the proof of concept is to create a barebones system that highlights the main functionalities required by the system at the simplest level. By the presentation date and time, the team should have a system complete enough to present a demonstration of at least one business use case from end to end as defined in the SRS document.
\\
\\ This will be achieved by building a simple web application that allows the user to create a form with very basic customization options such as creating questions and setting the form name. The created form should be sent to a backend for storage in a database and its details should be retreived by a mobile application. The mobile application should be able to display the forms contents to the user and allow them to fill it out and submit the responses back to the backend. Finally, the web application should retreive the submitted responses for the form and display some basic statistics about the data such as submission counts and question response distribution.

\section{Team Meeting Attendance}

\begin{table}[H]
\centering
\begin{tabular}{ll}
\toprule
\textbf{Student} & \textbf{Meetings}\\
\midrule
Total & 4\\
Ibrahim Quraishi & 4\\
Virochaan Ravichandran Gowri & 4\\
Mohammad Mahdi Mahboob & 4\\
Rayyan Suhail & 4\\
Omar Al-Asfar & 4\\
\bottomrule
\end{tabular}
\end{table}

\section{Supervisor/Stakeholder Meeting Attendance}

\noindent \textbf{Supervisor's Name: } Luke Schuurman

\begin{table}[H]
\centering
\begin{tabular}{ll}
\toprule
\textbf{Student} & \textbf{Meetings}\\
\midrule
Total & 2\\
Ibrahim Quraishi & 2\\
Virochaan Ravichandran Gowri & 2\\
Mohammad Mahdi Mahboob & 1\\
Rayyan Suhail & 2\\
Omar Al-Asfar & 2\\
\bottomrule
\end{tabular}
\end{table}

\section{Lecture Attendance}

\begin{table}[H]
\centering
\begin{tabular}{ll}
\toprule
\textbf{Student} & \textbf{Lectures}\\
\midrule
Total & 8\\
Ibrahim Quraishi & 7\\
Virochaan Ravichandran Gowri & 3\\
Mohammad Mahdi Mahboob & 3\\
Rayyan Suhail & 3\\
Omar Al-Asfar & 3\\
\bottomrule
\end{tabular}
\end{table}

This table includes all lectures attended from week 3 onwards when the team was formed. Ibrahim attended more lectures as it was later agreed by the team to have one person attend and relay information to save time.

\section{TA Document Discussion Attendance}

\noindent \textbf{TA's Name: } Tiago de Moraes Machado 

\begin{table}[H]
\centering
\begin{tabular}{ll}
\toprule
\textbf{Student} & \textbf{Lectures}\\
\midrule
Total & 3\\
Ibrahim Quraishi & 3\\
Virochaan Ravichandran Gowri & 3\\
Mohammad Mahdi Mahboob & 3\\
Rayyan Suhail & 3\\
Omar Al-Asfar & 3\\
\bottomrule
\end{tabular}
\end{table}

\section{Commits}

\begin{table}[H]
\centering
\begin{tabular}{lll}
\toprule
\textbf{Student} & \textbf{Commits} & \textbf{Percent}\\
\midrule
Total & 264 & 100\% \\
Ibrahim Quraishi & 40 & 15.2\% \\
Virochaan Ravichandran Gowri & 93 & 35.2\% \\
Rayyan Suhail & 45 & 17.0\% \\
Mohammad Mahdi Mahboob & 51 & 19.3\% \\
Omar Al-Asfar & 35 & 13.3\% \\
\bottomrule
\end{tabular}
\end{table}

The contribution percentage for Virochaan is much higher due to them handling most of the branch merging and conflict fixing when resolving pull requests, as well as differences in commit sizes.

\section{Issue Tracker}

\begin{table}[H]
\centering
\begin{tabular}{lll}
\toprule
\textbf{Student} & \textbf{Authored (O+C)} & \textbf{Assigned (C only)}\\
\midrule
Ibrahim Quraishi & 8 &  \\
Virochaan Ravichandran Gowr & 30 &  \\
Rayyan Suhail & 4 &  \\
Mohammad Mahdi Mahboob & 18 &  \\
Omar Al-Asfar & 6 &  \\
\bottomrule
\end{tabular}
\end{table}

The counts for some members are higher due to them taking on the role of project manager during certain phases, which involves creating and assigning more issues to team members. All members would discuss the creation and assignment of issues during meetings, and a certain member would take responsibility for creating them on GitHub (generally Virochaan as the repository owner). We also This was done intentionally to ensure a consistent format across the project board. As a result, the number of issues authored does not reflect the actual contributions of each member in this regard.\\\\
In addition, the team did not settle on a strict level of detail per issue. Some members preferred to break down tasks into smaller issues, while others opted to encompass all relevant tasks into a single issue. This also contributed to discrepancies in the number of issues authored by each member.

\section{CICD}

The main use of CI/CD for the duration of the project will be for regression testing. 
Every created unit test that has been verified to be correct will be added to the pipeline to ensure the correctness of the system remains the same between changes.
Additionally, the CI/CD will be used to enforce the AirBNB Javascript and JSX styling guides, as outlined in the development plan.

\section{Team Charter Trigger Items}

The team charter is outlined in the Development Plan, where various quantifiable triggers are defined to monitor team progress. These triggers are used to monitor important elements of team performance, including attendance, communication, and quality:
\begin{itemize}
    \item \textbf{Attendance:} All members are expected to attend meetings punctually for the entire agreed upon duration. Repeated unexcused absences constitue a trigger violation.
    \item \textbf{Communication:} All members are expected to respond to team communications within 24 hours. Failure to respond within this timeframe is considered a trigger violation. Moreover, failure to communicate scheduling difficulties or absences in advance is also considered a communication violation.
    \item \textbf{Quality:} All members are expected to complete their assigned tasks to an agreed upon standard. Repeated and blatant lack of effort or quality in completed tasks is considered a trigger violation. Members are expected to seek help from the team if they are struggling to complete their tasks to the expected standard. Failure to do so reflects on the entire team and is therefore a quality violation.
    \item \textbf{Attitude:} All members are expected to maintain a respectful and open-minded attitude towards each other. A good attitude facilitates effective teamwork and communication. Repeated negative or disrespectful behaviour is considered a trigger violation. Differences of opinion and conflict are not excusable under this trigger. All members are expected to address any issues maturely and professionally based on the team charter guidelines.
    \item \textbf{Contribution:} All members are expected to contribute equitably to the workload. Members are expected to proactively seek out tasks and responsibilities to support the team. Failure to achieve agreed upon performance metrics is a trigger violation. These expectations differ depending on the task, and may include number of commits, software features, lines of code, or even idea generation. The team will agree upon these metrics and expectations at the start of each phase.
\end{itemize}
No members have behaved in a manner that violates any of the team charter triggers to date. Everyone has generally met the aforementioned expectations, effectively collaborating and communicating to ensure the successful completion of deliverables.

\section{Additional Productivity Metrics}

\begin{itemize}
    \item \textbf{Pull Requests:} Highlight the number of pull requests opened by each member. This is a better indicator of contribution than commits alone, as it is compatible with our branching system. Number of commits can be misleading, as some members may make many small commits, while others make fewer but larger commits. Also, when a member merges from another branch it may appear as they commited all of that branches underlying commits. With pull requests, the real source of the contributions is clear. This metric will not be succeptible to inflation since we have agreed on a standard for pull requests that requires them to be of a reasonable size and scope.
    \item \textbf{Task Timeliness:} Highlight the number of days before the deadline each member completes their assigned tasks. This metric encourages members to complete their work ahead of schedule, allowing time for review and integration. Members who consistently complete tasks early can be assigned the foundational tasks in cases where other tasks are reliant on their completion.
\end{itemize}

\end{document}