\documentclass{article}

\usepackage{float}
\restylefloat{table}

\usepackage{booktabs}

\title{Team Contributions: POC\\\progname}

\author{\authname}

\date{}

\input{../Comments}
%% Common Parts

\newcommand{\progname}{Software Engineering} % PUT YOUR PROGRAM NAME HERE
\newcommand{\authname}{Team 4, EventHub
\\ Virochaan Ravichandran Gowri
\\ Omar Al-Asfar
\\ Rayyan Suhail
\\ Ibrahim Quraishi
\\ Mohammad Mahdi Mahboob} % AUTHOR NAMES                  

\usepackage{hyperref}
    \hypersetup{colorlinks=true, linkcolor=blue, citecolor=blue, filecolor=blue,
                urlcolor=blue, unicode=false}
    \urlstyle{same}
                                


\begin{document}

\maketitle

This document summarizes the contributions of each team member up to the POC
Demo.  The time period of interest is the time between the beginning of the term
and the POC demo.

\section{Demo Plans}

\wss{What will you be demonstrating}

\section{Team Meeting Attendance}

\wss{For each team member how many team meetings have they attended over the
time period of interest.  This number should be determined from the meeting
issues in the team's repo.  The first entry in the table should be the total
number of team meetings held by the team.}

\begin{table}[H]
\centering
\begin{tabular}{ll}
\toprule
\textbf{Student} & \textbf{Meetings}\\
\midrule
Total & 4\\
Ibrahim Quraishi & 4\\
Virochaan Ravichandran Gowri & 4\\
Mohammad Mahdi Mahboob & 4\\
Rayyan Suhail & 4\\
Omar Al-Asfar & 4\\
\bottomrule
\end{tabular}
\end{table}

\section{Supervisor/Stakeholder Meeting Attendance}

\wss{For each team member how many supervisor/stakeholder team meetings have
they attended over the time period of interest.  This number should be determined
from the supervisor meeting issues in the team's repo.  The first entry in the
table should be the total number of supervisor and team meetings held by the
team.  If there is no supervisor, there will usually be meetings with
stakeholders (potential users) that can serve a similar purpose.}

\noindent \textbf{Supervisor's Name: } [fill in this information]

\begin{table}[H]
\centering
\begin{tabular}{ll}
\toprule
\textbf{Student} & \textbf{Meetings}\\
\midrule
Total & 2\\
Ibrahim Quraishi & 2\\
Virochaan Ravichandran Gowri & 2\\
Mohammad Mahdi Mahboob & 1\\
Rayyan Suhail & 2\\
Omar Al-Asfar & 2\\
\bottomrule
\end{tabular}
\end{table}

\wss{If needed, an explanation for the counts can be provided here.}

\section{Lecture Attendance}

\wss{NOTE: There will be approximately 13 lectures between the start of class
and the POC demos}

\begin{table}[H]
\centering
\begin{tabular}{ll}
\toprule
\textbf{Student} & \textbf{Lectures}\\
\midrule
Total & 8\\
Ibrahim Quraishi & 7\\
Virochaan Ravichandran Gowri & 3\\
Mohammad Mahdi Mahboob & 3\\
Rayyan Suhail & 3\\
Omar Al-Asfar & 3\\
\bottomrule
\end{tabular}
\end{table}

This table includes all lectures attended from week 3 onwards when the team was formed. Ibrahim attended more lectures as it was later agreed by the team to have one person attend and relay information to save time.

\section{TA Document Discussion Attendance}

\wss{For each team member how many of the informal document discussion meetings
with the TA were attended over the time period of interest.}

\noindent \textbf{TA's Name: } [Tiago de Moraes Machado]

\begin{table}[H]
\centering
\begin{tabular}{ll}
\toprule
\textbf{Student} & \textbf{Lectures}\\
\midrule
Total & 3\\
Ibrahim Quraishi & 3\\
Virochaan Ravichandran Gowri & 3\\
Mohammad Mahdi Mahboob & 3\\
Rayyan Suhail & 3\\
Omar Al-Asfar & 3\\
\bottomrule
\end{tabular}
\end{table}

\section{Commits}

\wss{For each team member how many commits to the main branch have been made
over the time period of interest.  The total is the total number of commits for
the entire team since the beginning of the term.  The percentage is the
percentage of the total commits made by each team member.}

\begin{table}[H]
\centering
\begin{tabular}{lll}
\toprule
\textbf{Student} & \textbf{Commits} & \textbf{Percent}\\
\midrule
Total & 264 & 100\% \\
Ibrahim Quraishi & 40 & 15.2\% \\
Virochaan Ravichandran Gowri & 93 & 35.2\% \\
Rayyan Suhail & 45 & 17.0\% \\
Mohammad Mahdi Mahboob & 51 & 19.3\% \\
Omar Al-Asfar & 35 & 13.3\% \\
\bottomrule
\end{tabular}
\end{table}

The contribution percentage for Virochaan is much higher due to them handling most of the branch merging and conflict fixing when resolving pull requests, as well as differences in commit sizes.

\section{Issue Tracker}

\wss{For each team member how many issues have they authored (including open and
closed issues (O+C)) and how many have they been assigned (only counting closed
issues (C only)) over the time period of interest.}

\begin{table}[H]
\centering
\begin{tabular}{lll}
\toprule
\textbf{Student} & \textbf{Authored (O+C)} & \textbf{Assigned (C only)}\\
\midrule
Name 1 & Num & Num \\
Name 2 & Num & Num \\
Name 3 & Num & Num \\
Name 4 & Num & Num \\
Name 5 & Num & Num \\
\bottomrule
\end{tabular}
\end{table}

\wss{If needed, an explanation for the counts can be provided here.}

\section{CICD}

The main use of CI/CD for the duration of the project will be for regression testing. 
Every created unit test that has been verified to be correct will be added to the pipeline to ensure the correctness of the system remains the same between changes.
Additionally, the CI/CD will be used to enforce the AirBNB Javascript and JSX styling guides, as outlined in the development plan.

\section{Team Charter Trigger Items}

\wss{Provide a summary of the quantified triggers identified in the team's
charter.}

\wss{Provide a list of any violations of the triggers.  If the team wishes, the
violations can be summarized on aggregate, instead of naming specific team
members.}

\wss{Provide a plan to address the violations.  This could include revising the
triggers, if they are found to be too weak, strong or ambiguous.}

\section{Additional Productivity Metrics}

\wss{If your team has additional metrics of productivity, please feel free to
add them to this report.}

\end{document}