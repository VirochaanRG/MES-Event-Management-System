\subsubsection*{Virochaan Ravichandran Gowri Reflection:}
\begin{enumerate}
  \item \textbf{What went well while writing this deliverable? 
  During the writing} \\
  During the writing of this deliverable we all had a clear understanding of the goals of this project and the general plan which we did in the past deliverable. From this through conversations with our supervisors we were easily able to obtain requirements that were relevant to our project. The best part was the communication between the members of our team and with the supervisor. This allowed us to 
  \item \textbf{What pain points did you experience during this deliverable, and how did you resolve them?} \\
  We needed to make sure that we remained flexible in our requirements if we had to make changes in the future. This was especially crucial because we knew that there could be a chance that there could be additional requirements or changes based on what the client wanted. Also the client wanted a certain stack so we didn't really have flexibility regarding that so it was some new technologies that we need to become accustomed with. 
  \item \textbf{For each of the knowledge areas and skills identified in the previous question, what are at least two approaches to acquiring the knowledge or mastering the skill?  Of the identified approaches, which will each team member pursue, and why did they make this choice?}
\end{enumerate}

\subsubsection*{Rayyan Suhail Reflection:}
\begin{enumerate}
  \item \textbf{What went well while writing this deliverable?} \\
  I worked on the Purpose, Relevant Facts and Assumptions, New Problems, Tasks, Migration, and User Documentation sections. What went well was how everything started connecting once we finalized our project direction. Writing the User Documentation section was interesting since we focused on in-app guidance instead of long manuals. The planning sections also came together nicely after a few team discussions, and overall, our communication made the process much smoother.

  \item \textbf{What pain points did you experience during this deliverable, and how did you resolve them?} \\
  The main challenge was keeping the document consistent and not too repetitive across similar sections. Some template areas, like Migration and New Problems, were tricky to adapt to our system at first. I resolved this by reviewing past SRS examples and working closely with my teammates to make sure our sections aligned and stayed concise.

  \item \textbf{For each of the knowledge areas and skills identified in the previous question, what are at least two approaches to acquiring the knowledge or mastering the skill? Of the identified approaches, which will each team member pursue, and why did they make this choice?} \\
  To improve my documentation and planning skills, I plan to:
  \begin{itemize}
    \item Review more example SRS documents to better understand structure and tone.  
    \item Keep practicing LaTeX and version control to make editing and collaboration smoother.
  \end{itemize}
  I’ll also focus on improving coordination by:
  \begin{itemize}
    \item Staying synced with the backend and mobile teams on task priorities.  
    \item Using GitHub Projects or Notion to track progress and deadlines.
  \end{itemize}
\end{enumerate}





\subsubsection*{Group Reflection}
\begin{enumerate}
  \item How many of your requirements were inspired by speaking to your client(s) or their proxies (e.g. your peers, stakeholders, potential users)?
  Most of our functional requirements were obtained directly through conversations with out supervisor (Luke) or through todcumentation provided by him. This provided us with a good starting point on what his vision of the system would look like 

  \item Which of the courses you have taken, or are currently taking, will help your team to be successful with your capstone project.
  \begin{itemize}
      \item SFWRENG 4HC3: This course will help us design the frontend components and create a platform which will provide a positive user experience.
      \item SFWRENG 3DB3: Help us design effective database schema and queries.
      \item SFWRENG 3A04: This will help us design our architecture for the system.
  \end{itemize}

  \item What knowledge and skills will the team collectively need to acquire to successfully complete this capstone project?  Examples of possible knowledge to acquire include domain specific knowledge from the domain of your application, or software engineering knowledge, mechatronics knowledge or computer science knowledge.  Skills may be related to technology, or writing, or presentation, or team management, etc. You should look to identify at
  least one item for each team member.
\end{enumerate}
