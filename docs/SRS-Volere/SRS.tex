% THIS DOCUMENT IS FOLLOWS THE VOLERE TEMPLATE BY Suzanne Robertson and James Robertson
% ONLY THE SECTION HEADINGS ARE PROVIDED
%
% Initial draft from https://github.com/Dieblich/volere
%
% Risks are removed because they are covered by the Hazard Analysis
\documentclass[12pt]{article}

\usepackage{booktabs}
\usepackage{tabularx}
\usepackage{hyperref}
\usepackage{graphicx}
\usepackage{caption}
\usepackage{float}
\usepackage{enumitem}

\hypersetup{
    bookmarks=true,         % show bookmarks bar?
      colorlinks=true,      % false: boxed links; true: colored links
    linkcolor=red,          % color of internal links (change box color with linkbordercolor)
    citecolor=green,        % color of links to bibliography
    filecolor=magenta,      % color of file links
    urlcolor=cyan           % color of external links
}

\newcommand{\lips}{\textit{Insert your content here.}}

\input{../Comments}
%% Common Parts

\newcommand{\progname}{Software Engineering} % PUT YOUR PROGRAM NAME HERE
\newcommand{\authname}{Team 4, EventHub
\\ Virochaan Ravichandran Gowri
\\ Omar Al-Asfar
\\ Rayyan Suhail
\\ Ibrahim Quraishi
\\ Mohammad Mahdi Mahboob} % AUTHOR NAMES                  

\usepackage{hyperref}
    \hypersetup{colorlinks=true, linkcolor=blue, citecolor=blue, filecolor=blue,
                urlcolor=blue, unicode=false}
    \urlstyle{same}
                                


\begin{document}

\title{Software Requirements Specification for \progname: subtitle describing software} 
\author{\authname}
\date{\today}
	
\maketitle

~\newpage

\pagenumbering{roman}

\tableofcontents

~\newpage

\section*{Revision History}

\begin{tabularx}{\textwidth}{p{3cm}p{2cm}X}
\toprule {\textbf{Date}} & {\textbf{Version}} & {\textbf{Notes}}\\
\midrule
Date 1 & 1.0 & Notes\\
Date 2 & 1.1 & Notes\\
\bottomrule
\end{tabularx}

~\\

~\newpage
\section{Purpose of the Project}
\subsection{User Business}
\lips
\subsection{Goals of the Project}
\lips
\section{Stakeholders}
\subsection{Client}
\lips
\subsection{Customer}
\lips
\subsection{Other Stakeholders}
\lips
\subsection{Hands-On Users of the Project}
\lips
\subsection{Personas}
\lips
\subsection{Priorities Assigned to Users}
\lips
\subsection{User Participation}
\lips
\subsection{Maintenance Users and Service Technicians}
\lips

\section{Mandated Constraints}
\subsection{Solution Constraints}
\lips
\subsection{Implementation Environment of the Current System}
\lips
\subsection{Partner or Collaborative Applications}
\lips
\subsection{Off-the-Shelf Software}
\lips
\subsection{Anticipated Workplace Environment}
\lips
\subsection{Schedule Constraints}
\lips
\subsection{Budget Constraints}
\lips
\subsection{Enterprise Constraints}
\lips

\section{Naming Conventions and Terminology}
\subsection{Glossary of All Terms, Including Acronyms, Used by Stakeholders
involved in the Project}
\lips

\section{Relevant Facts And Assumptions}
\subsection{Relevant Facts}
\lips
\subsection{Business Rules}
\lips
\subsection{Assumptions}
\lips

\section{The Scope of the Work}

The purpose of the section is to define the expected scope of the project, including specifications on how the system is partitioned, business data models, and business use cases.

\subsection{The Current Situation}

Currently, the MES uses plethora of different platforms to host events and conduct surveys such as tools for registration, signing of waivers, and checking in attendees. The main platforms used are a combination of Google Forms and Google Sheets. There are several pain points to address with the current system.

\begin{itemize}
  \item \textbf{Decentralized System Components} Registration and surverying is split across a plethora of tools and software which may not be compatible with each other. For example, Google Forms is used for registration forms and surveys, Google Sheets stores form data and LinkTree holds links to all the registration forms, and event updates is done by email or social media. 
  \item \textbf{Overly Complex Form Logic} The current CFES survey consists of 70 pages of questions linked together through complex branching logic. The Google Forms form building UI makes this very complicated as all form elements are displayed as a linear list of sections making it very hard to track paths through the form.
  \item \textbf{Disorganized Data Visualization} Form response data is currently stored using Google Sheets. While Google Form data is easily imported into Google Sheets, any analytics on the data must be done manually through equations and macros.
  \item \textbf{Lack of Reusability} Google Forms comes with a few templates that provide an initial starting point for many types of forms such as registrant information. However, after the first section of the form, each subsequent section must be made manually. Sections may be imported from other forms but this requires the user to have access to a form with the wanted section and to scour through an unorganized list of forms and sections.
  \item \textbf{Low Response Rates on Surveys} The response rates on the annual CFES survey of undergraduate engineering students have been decreasing due to the long length of the Google Form and the lack of ability to submit a partially completed form.
  \item \textbf{Manual Registration Scheduling} Event registration is managed through a combination of Instragram, Google Forms, and LinkTree. Events are advertised on Instagram, and a link to the signup Google Form is posted on the MES LinkTree. This solution lacks automation since a new Google Form must be made for every event and links have to be manually added and removed for each form when registration is opened or closed. 
\end{itemize}
% MES hosts lots of events per year
% Their event management system is unorganized
% Consists of google forms, google sheets
% No template form to work with redundant info between forms like name, number email, etc
% Lots of steps to registration and split across too many platforms
% The same thing is done for the annual CFES survey of undergrad eng students
% long 70 page form of complex branching logic, low response rate
% 

\subsection{The Context of the Work}

This section provides an overview of the high-level inputs and outputs between the system and external systems or actors. Figure X illustrates a work context diagram showing the interactions between the system components and external elements.

\begin{center}
\begin{figure}[H]
    \centering
    \includegraphics[width=1\linewidth]{images/work_context.png}
    \caption{Work context diagram of the form builder and event hosting system}\label{workcontext}
\end{figure}
\end{center}

\subsection{Work Partitioning}
% split the workflow of the entire system into sub-workflows

This section describes how the work done by the system can be partitioned into smaller and more manageable workflows. Below is a table describing each of the sub-workflows of the proposed systems. Note that event creation and survey creation have been grouped as a single workflow since they are nearly identical.

{\renewcommand{\arraystretch}{1.2}
\begin{table}[H]
\centering
\begin{tabular}{|l|l|l|l|l|}
\hline
Event No. & Event                               & Input                & Output                                                                        & Requirements \\ \hline
1         & User registers for event            & Registration data    & Event ticket                                                                  &              \\ \hline
2         & User fills out a survey             & Survey responses     & Survey completion confirmation                                                &              \\ \hline
3         & Admin creates a form module         & Module question data & Reusable form module                                                          &              \\ \hline
4         & Admin creates an event/survey       & Event/survey data    & Event/survey creation confirmation                                            &              \\ \hline
5         & Admin views survey/event statistics & Event/survey to view & Event/survey registration/response reports                                    &              \\ \hline
\end{tabular}
\captionof{table}{Partitioning of system workflows}\label{wfpart}
\end{table}
}

\subsection{Specifying a Business Use Case (BUC)}
% provide a use case for each sub workflow

\noindent\textbf{BUC 1: } User registers for an event \\
\textbf{Input:} Registration data \\
\textbf{Output:} Event tickets \\
\textbf{Pre-condition:} User has downloaded the application and has created an account \\
\textbf{Scenario:} \\
\begin{enumerate}
  \item The user application receives the registration data
  \item The user application verifies the data is filled correctly
  \item The user application sends the registration data to the backend server
  \item The backend server verifies the event is not full and the deadline has not passed
  \item The backend server adds the user to the list of attendees
  \item The backend server generates an event ticket and sends it to the user application
  \item The user application confirms with the user that the registration was successful and presents the user with the ticket
\end{enumerate}
\textbf{Sub variations:} \\
2a. The submitted registration data has errors, the user application prompts the user to fix the errors and resubmit \\
4a. The event is full or the deadline has passed, the backend sends an error code to the user application \\
4b. The user application alerts the user of the error. \\

\noindent\textbf{BUC 2: } User fills out a survey module \\
\textbf{Input:} Survey responses \\
\textbf{Output:} Survey completion confirmation \\
\textbf{Pre-condition:} User has downloaded the application and has created an account \\
\textbf{Scenario:} \\
\begin{enumerate}
  \item The user application receives the survey data
  \item The user application verifies the data is filled correctly
  \item The user application sends the survey data to the backend server
  \item The backend server updates the survey database with the user’s data
  \item The backend server sends a confirmation message to the user application
  \item The user application confirms with the user that the survey data has been submitted
\end{enumerate}
\textbf{Sub variations:} \\
2a. The submitted data has errors (i.e. mandatory fields not filled), the user application prompts the user to fix the errors and resubmit \\

\noindent\textbf{BUC 3: } Admin creates a form module \\
\textbf{Input:} Form fields \\
\textbf{Output:} Reusable form module \\
\textbf{Pre-condition:} Admin has access to create custom form modules \\
\textbf{Scenario:} \\
\begin{enumerate}
  \item The admin portal receives the list of form fields and questions from the admin user for the custom module
  \item The admin portal verifies the custom module has been created correctly
  \item The admin portal sends the custom module data to the backend server
  \item The backend server authenticates the admin user
  \item The backend server saves the custom module data to the template database
  \item The backend server sends a confirmation message to the admin portal
  \item The admin portal updates the list of custom modules with the completed module
  \item The admin portal confirms with the admin user that the custom module has been created
\end{enumerate}
\textbf{Sub variations:} \\
2a. The submitted form module has errors (i.e. unfinished fields), the admin user is prompted to fix these errors before resubmitting \\
4a. Authentication of the admin fails, the admin portal is notified of the request denial \\

\noindent\textbf{BUC 4: } Admin creates an event/survey \\
\textbf{Input:} Event details \\
\textbf{Output:} Event creation confirmation \\
\textbf{Pre-condition:} Admin has access to create events \\
\textbf{Scenario:} \\
\begin{enumerate}
  \item The admin portal receives the event/survey details
  \item The admin portal verifies the event/survey details are correct
  \item The admin portal sends the event/survey data to the backend server
  \item The backend server authenticates the admin user
  \item The backend server saves the event/survey data to the database of events/surveys
  \item The backend server sends a message to the user application to notify users of the new event/survey
  \item The backend server sends a confirmation message to the admin portal
  \item The admin portal adds the created event/survey to the event/survey dashboard
  \item The admin portal confirms with the admin user that the event/survey has been created
\end{enumerate}
\textbf{Sub variations:} \\
2a. The submitted details have errors (i.e. event date has already passed), the admin user is prompted to fix these errors before resubmitting \\
4a. Authentication of the admin fails, the admin portal is notified of the request denial \\

\noindent\textbf{BUC 5: } Admin views event/survey statistics \\
\textbf{Input:} Event/survey to view \\
\textbf{Output:} Event registration/survey response report \\
\textbf{Pre-condition:} Event/survey has been created, and users have registered/responded \\
\textbf{Scenario:} \\
\begin{enumerate}
  \item The admin portal receives the request to view event/survey statistics
  \item The admin portal sends a request to the backend server containing the identification for the event/survey to view
  \item The backend server receives the request and authenticates the admin user
  \item The backend server retrieves the registrant/response data for the requested event/survey from the database
  \item The backend server generates a statistical report of all the data
  \item The data is sent back to the admin portal
  \item The admin portal formats all the data into a readable format
  \item The admin user is presented with the event/survey statistics
\end{enumerate}
\textbf{Sub variations:} \\
3a. Authentication of the admin fails, the admin portal is notified of the request denial \\


\section{Business Data Model and Data Dictionary}
\subsection{Business Data Model}
\lips
% Split the entire system into subsystems which the data passes through and the form it takes at each step 
% This is NOT the same as workflow partition. This deals with DATA and the other deals with PROCESSES
\subsection{Data Dictionary}
\lips
% Definition of each subsystem defined above
\section{The Scope of the Product}
\subsection{Product Boundary}
\lips
% Produce a Product Boundary Diagram which shows what happens inside of the system and how it happens, and what comes from/goes outside the system
\subsection{Product Use Case Table}
\lips
% Label each product use case from the diagram and map them to actors, inputs/outputs, and requirement numbers
\subsection{Individual Product Use Cases (PUC's)}
\lips
% Take each Product user case and define it further with the following
% Trigger - what caused this use case to execute
% Actors - what actors caused the trigger
% Preconditions - what properties hold before the use case occurs
% Input - input to the use case
% output - output of the use case
% outcome - What happened as a result of the use case
\section{Functional Requirements}
\subsection{Functional Requirements}
\lips

\section{Look and Feel Requirements}
\subsection{Appearance Requirements}
\begin{enumerate}[label=LFR-AP.\arabic*, wide=0pt, leftmargin=*]
  \item The interface shall comply with MES branding criteria (logo, colour scheme).\\[2mm]
    {\bf Motivation:} Ensures that the product is recognizable as an official MES event registration platform.\\
    {\bf Fit Criterion:} The MES representative shall certify that the product complies with the current standards.
  \item The tool shall have a clean, minimalist layout prioritizing the functionality.\\[2mm]
    {\bf Motivation:} Increases accessibility and user productivity, and allows for consistency across different events.\\
    {\bf Fit Criterion:} During usability testing across the MES events, at least 80\% of students rate the forums ease of use a 4+.
  \item Admin dashboard shall clearly display relevant analytics, and distinct charts.\\[2mm]
    {\bf Motivation:} Improves organizer accessibility and readability, allowing for easier management during planning operations.\\
    {\bf Fit Criterion:} The representative shall approve of the analytics layout.
\end{enumerate}

\subsection{Style Requirements}
\begin{enumerate}[label=LFR-S.\arabic*, wide=0pt, leftmargin=*]
  \item The design shall be professional but approachable, accomodating all types of events.\\[2mm]
    {\bf Motivation:} The app should have a friendly vibe for casual student use, but maintain a sense of maturity for the more formal uses.\\
    {\bf Fit Criterion:} After use in some MES events, 70\% of users shall agree that they trust and respect the product, and 70\% shall agree that it was not boring.
  \item The admin dashboard shall emphasize functionality and clarity, avoiding distractions just for the sake of aesthetics.\\[2mm]
    {\bf Motivation:} Admin and organizers require clear data and analytics, they have no use for stylistic components.\\
    {\bf Fit Criterion:} A sample of event organizers and club executives approve and are able to interpret all analytics without prior explanation.
\end{enumerate}

\section{Usability and Humanity Requirements}
\subsection{Ease of Use Requirements}
\begin{enumerate}[label=UHR-EoU.\arabic*, wide=0pt, leftmargin=*]
  \item Registration and feedback forms shall be easily and quickly filled out for anyone with no instruction.\\[2mm]
    {\bf Motivation:} The app should have a friendly vibe for casual student use, but maintain a sense of maturity for the more formal uses.\\
    {\bf Fit Criterion:} After use in some MES events, 70\% of users shall agree that they trust and respect the product, and 70\% shall agree that it was not boring.
  \item Users shall sign-up/sign-in to the app with no instruction. \\[2mm]
    {\bf Motivation:} The registration process is the most important aspect of the platform, so the app should have a clear and simple layout.\\
    {\bf Fit Criterion:} A sample of users shall register/login within 2 minutes without instruction.
  \item The admin dashboard shall emphasize functionality and clarity, avoiding distractions just for the sake of aesthetics.\\[2mm]
    {\bf Motivation:} Admin and organizers require clear data and analytics, they have no use for stylistic components.\\
    {\bf Fit Criterion:} A sample of event organizers and club executives approve and are able to interpret all analytics without prior explanation.
\end{enumerate}

\subsection{Personalization and Internationalization Requirements}
\begin{enumerate}[label=UHR-PIR.\arabic*, wide=0pt, leftmargin=*]
  \item The custom form builder shall allow admin to create personalized forms with event-specific fields.\\[2mm]
    {\bf Motivation:} Flexibility is essential to accomodate the diverse set of events as per the scope.\\
    {\bf Fit Criterion:} The forum is used for 2 events successfully without developer assistance.
  \item Admin dashboard shall support sorting and filtering of attendees by predefined metrics.\\[2mm]
    {\bf Motivation:} Streamlines the attendee organizing process, allowing admin to categorize attendees as per the event requirements.\\
    {\bf Fit Criterion:} Organizers shall filter attendees within 1 minute of use without instruction.
\end{enumerate}

\subsection{Learning Requirements}
\begin{enumerate}[label=UHR-LR.\arabic*, wide=0pt, leftmargin=*]
  \item The app shall be easy for anyone with an intermediate level of English.\\[2mm]
    {\bf Motivation:} Users should expect intuitive, concise ineteractions.\\
    {\bf Fit Criterion:} 80\% of students successfully complete a form with no assistance.
  \item Admin shall be able to build a form with the custom builder with no prior instruction.\\[2mm]
    {\bf Motivation:} Event organizers are constantly changing, so it is important to reduce dependency on technical training.\\
    {\bf Fit Criterion:} 80\% of organizers shall complete a basic test form on their first attempt.
\end{enumerate}

\subsection{Understandability and Politeness Requirements}
\begin{enumerate}[label=UHR-UPR.\arabic*, wide=0pt, leftmargin=*]
  \item Basic, non-technical language will be used (Sign-up instead of RSVP).\\[2mm]
    {\bf Motivation:} Increases accessibility among non-technical users, or non-fluent English speakers.\\
    {\bf Fit Criterion:} 80\% of students understand all wording without clarification.
  \item The app shall communicate errors clearly and politely.\\[2mm]
    {\bf Motivation:} Clear error messages reduce frustration and improve user experience.\\
    {\bf Fit Criterion:} Feedback suggests that 80\% of students found the experience satisfying.
  \item The app shall hide sensitive and confidential information from users.\\[2mm]
    {\bf Motivation:} Enhances security and confidence of organizations in the product.\\
    {\bf Fit Criterion:} Approval from MES representative.
\end{enumerate}

\subsection{Accessibility Requirements}
\lips

\section{Performance Requirements}
\subsection{Speed and Latency Requirements}
\lips
\subsection{Safety-Critical Requirements}
\lips
\subsection{Precision or Accuracy Requirements}
\lips
\subsection{Robustness or Fault-Tolerance Requirements}
\lips
\subsection{Capacity Requirements}
\lips
\subsection{Scalability or Extensibility Requirements}
\lips
\subsection{Longevity Requirements}
\lips

\section{Operational and Environmental Requirements}
\subsection{Expected Physical Environment}
\lips
\subsection{Wider Environment Requirements}
\lips
\subsection{Requirements for Interfacing with Adjacent Systems}
\lips
\subsection{Productization Requirements}
\lips
\subsection{Release Requirements}
\lips

\section{Maintainability and Support Requirements}
\subsection{Maintenance Requirements}
\lips
\subsection{Supportability Requirements}
\lips
\subsection{Adaptability Requirements}
\lips

\section{Security Requirements}
\subsection{Access Requirements}
\lips
\subsection{Integrity Requirements}
\lips
\subsection{Privacy Requirements}
\lips
\subsection{Audit Requirements}
\lips
\subsection{Immunity Requirements}
\lips

\section{Cultural Requirements}
\subsection{Cultural Requirements}
\lips

\section{Compliance Requirements}
\subsection{Legal Requirements}
\lips
\subsection{Standards Compliance Requirements}
\lips

\section{Open Issues}
\lips
% Any problems that remain unresolved and undefined 
\section{Off-the-Shelf Solutions}
\subsection{Ready-Made Products}
\lips
\subsection{Reusable Components}
\lips
\subsection{Products That Can Be Copied}
\lips

\section{New Problems}
\subsection{Effects on the Current Environment}
\lips
\subsection{Effects on the Installed Systems}
\lips
\subsection{Potential User Problems}
\lips
\subsection{Limitations in the Anticipated Implementation Environment That May
Inhibit the New Product}
\lips
\subsection{Follow-Up Problems}
\lips

\section{Tasks}
\subsection{Project Planning}
\lips
\subsection{Planning of the Development Phases}
\lips

\section{Migration to the New Product}
\subsection{Requirements for Migration to the New Product}
\lips
\subsection{Data That Has to be Modified or Translated for the New System}
\lips

\section{Costs}
\lips
\section{User Documentation and Training}
\subsection{User Documentation Requirements}
\lips
\subsection{Training Requirements}
\lips

\section{Waiting Room}
\lips

\section{Ideas for Solution}
\lips

\newpage{}
\section*{Appendix --- Reflection}

\input{../Reflection.tex}

\subsubsection*{Virochaan Ravichandran Gowri Reflection}
\begin{enumerate}
  \item \textbf{What went well while writing this deliverable?} \\
  During the writing of this deliverable we all had a clear understanding of the goals of this project and the general
  plan which we did in the past deliverable. From this through conversations with our supervisors we were easily able to
  obtain requirements that were relevant to our project. The best part was the communication between the members of our
  team and with the supervisor. This allowed us to
  \item \textbf{What pain points did you experience during this deliverable, and how did you resolve them?} \\
  We needed to make sure that we remained flexible in our requirements if we had to make changes in the future. This was
  especially crucial because we knew that there could be a chance that there could be additional requirements or changes
  based on what the client wanted. Also the client wanted a certain stack so we didn't really have flexibility regarding
  that so it was some new technologies that we need to become accustomed with.

\end{enumerate}

\subsubsection*{Mohammad Mahdi Mahboob Reflection}
\begin{enumerate}
  \item \textbf{What went well while writing this deliverable?} \\
  The projects constraints section write-up went smoothly. Due to previous meetings with the client as well as
  documentation outlined in previous deliverables, much of the constraints were readily available and thereby easily
  articulable. Most constraints regarding the environment of the solution as well as specific requests were
  well-documented in our communications with the client; their technical aptitude was greatly appreciated since they
  were able to provide us with a clear technology and development roadmap to facilitate collaboration across Capstone
  groups for the integrated application.
  \item \textbf{What pain points did you experience during this deliverable, and how did you resolve them?} \\
  Some pain points I encountered during this deliverable were encountered when articulating the requirements. It was
  difficult to gauge what an appropriate level of granularity would be for some of the requirements, as well as decide
  whether requirements for some subsections were necessary. Defining fit criteria was also difficult at times,
  especially for requirements which related to the internal behaviour of the system, such as FR-11. In order to resolve
  these discrepancies, I consulted my teammates and brought forth these concerns during supervisor meetings to
  clarify details, as well as explicitly ascertain the relevance of different requirement categories, such as
  compliance. I would like to commend my teammates in this regard, as well as thank our supervisor for their time and
  expertise.
\end{enumerate}

\subsubsection*{Group Reflection}
\begin{enumerate}
  \item How many of your requirements were inspired by speaking to your client(s) or their proxies (e.g. your peers, stakeholders, potential users)?
  Most of our functional requirements were obtained directly through conversations with out supervisor (Luke) or through todcumentation provided by him. This provided us with a good starting point on what his vision of the system would look like and from there we built upon and added requiremnts based on what we throught the system needed.

  \item Which of the courses you have taken, or are currently taking, will help your team to be successful with your capstone project.
  \begin{itemize}
      \item SFWRENG 4HC3: This course will help us design the frontend components and create a platform which will provide a positive user experience.
      \item SFWRENG 3DB3: Help us design effective database schema and queries.
      \item SFWRENG 3A04: This will help us design our architecture for the system. Also gave us experience working with git control and designing sytem with diagrams.
      \item SFWRENG 2AA4: Designing full scale software and using github project management.
      \item ENG 3PX3 + 2PX3: This course gave us some basic project and team management skills as we had to work on semester long projects similar to what is being done in this course.
      \item SFWRENG 3RA3: Gave us help writing this document and will help us refine and update our requirements in the future.
  \end{itemize}

  \item \textbf{What knowledge and skills will the team collectively need to acquire to successfully complete this capstone project?  Examples of possible knowledge to acquire include domain specific knowledge from the domain of your application, or software engineering knowledge, mechatronics knowledge or computer science knowledge.  Skills may be related to technology, or writing, or presentation, or team management, etc. You should look to identify at least one item for each team member.}
  \begin{itemize}
    \item Full-Stack Web Development: Some skills in this category that will be needed to be learned include requiring the development of front-end
  \end{itemize}
  \item \textbf{For each of the knowledge areas and skills identified in the previous question, what are at least two approaches to acquiring the knowledge or mastering the skill?  Of the identified approaches, which will each team member pursue, and why did they make this choice?}
\end{enumerate}


\end{document}