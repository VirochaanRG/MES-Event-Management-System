% THIS DOCUMENT IS FOLLOWS THE VOLERE TEMPLATE BY Suzanne Robertson and James Robertson
% ONLY THE SECTION HEADINGS ARE PROVIDED
%
% Initial draft from https://github.com/Dieblich/volere
%
% Risks are removed because they are covered by the Hazard Analysis
\documentclass[12pt]{article}

\usepackage{booktabs}
\usepackage{tabularx}
\usepackage{hyperref}
\hypersetup{
    bookmarks=true,         % show bookmarks bar?
      colorlinks=true,      % false: boxed links; true: colored links
    linkcolor=red,          % color of internal links (change box color with linkbordercolor)
    citecolor=green,        % color of links to bibliography
    filecolor=magenta,      % color of file links
    urlcolor=cyan           % color of external links
}

\newcommand{\lips}{\textit{Insert your content here.}}

%% Comments

\usepackage{color}

\newif\ifcomments\commentstrue %displays comments
%\newif\ifcomments\commentsfalse %so that comments do not display

\ifcomments
\newcommand{\authornote}[3]{\textcolor{#1}{[#3 ---#2]}}
\newcommand{\todo}[1]{\textcolor{red}{[TODO: #1]}}
\else
\newcommand{\authornote}[3]{}
\newcommand{\todo}[1]{}
\fi

\newcommand{\wss}[1]{\authornote{magenta}{SS}{#1}} 
\newcommand{\plt}[1]{\authornote{cyan}{TPLT}{#1}} %For explanation of the template
\newcommand{\an}[1]{\authornote{cyan}{Author}{#1}}

%% Common Parts

\newcommand{\progname}{Software Engineering} % PUT YOUR PROGRAM NAME HERE
\newcommand{\teamname}{EvENGage}
\newcommand{\authname}{Team 4, \teamname
\\ Virochaan Ravichandran Gowri
\\ Omar Al-Asfar
\\ Rayyan Suhail
\\ Ibrahim Quraishi
\\ Mohammad Mahdi Mahboob} % AUTHOR NAMES

\newcommand{\prjdesc}{MES Event Management Registration, Administration, and Survey Analytics}

\usepackage{hyperref}
    \hypersetup{colorlinks=true, linkcolor=blue, citecolor=blue, filecolor=blue,
                urlcolor=blue, unicode=false}
    \urlstyle{same}



\begin{document}

\title{Software Requirements Specification for \progname: subtitle describing software} 
\author{\authname}
\date{\today}
	
\maketitle

~\newpage

\pagenumbering{roman}

\tableofcontents

~\newpage

\subsection*{Revision History}

\begin{tabularx}{\textwidth}{p{3cm}p{2cm}X}
\toprule {\textbf{Date}} & {\textbf{Version}} & {\textbf{Notes}}\\
\midrule
Date 1 & 1.0 & Notes\\
Date 2 & 1.1 & Notes\\
\bottomrule
\end{tabularx}

~\newpage
\section{Project Drivers}
\subsection{Purpose of the Project}
\subsubsection{User Business}
\lips
\subsubsection{Goals of the Project}
\lips
\subsection{Stakeholders}
\subsubsection{Client}
The client for this project is the supervisor, Luke Schuurman. He is a member of the MES and has first hand experience plannig and hosting events with the MES. As the supervisor he will play a crucial role by ensuring the project aligns with objective of the MES and integrate the platform seamlessly with existing systems. He will also provide us with feedback and guidance throughout the project and will help define the project requirements in this document.
\subsubsection{Customer}
Refer to client
\subsubsection{Other Stakeholders}
Irrelevant
\subsubsection{Hands-On Users of the Project}
\textbf{MES Executives and Council Members}: They will be utilizing this system to create and manage events, configure forms and surveys, monitor event data and generate data analytics reports. They can be characterized as primarily undergraduate students who value their time greatly. They aim to reduce the time taken to do administrative tasks as well as provide a better experience during MES events. Their experience with systems like this can range from Journeyman - Master as they can be experienced in event planning and student engagement. There may be a slight learning curve to utilizing the new technology but these users have experience performing these functions.\\
\newline
\textbf{McMaster Engineering Students}: They will be utilizing this app to to register for events, purchase tickets, sign waivers, check in at venues, and complete feedback surveys. They want to enjoy their university experience and connect with other students, They are also very busy and value their time greatly so are looking for an intuitive and straightforward user experience. Their general experience with systems like this is Journeyman as they may have used similar systems for other use cases. \\
\newline
\textbf{Other Students and Event Attendees}: This group includes non-engineering students, alumni, and invited guests who participate in large-scale MES events such as the Fireball Formal and Graduation Formal, which extend beyond the core engineering community. It could also include students and guests from other universities as well as industry which attend the engineering conferences that the MES helps host. They are generally looking for an easy and seamless experience registering and attending these events. Their general experience with systems like this is Journeyman as they may have used similar systems for other use cases.
\subsubsection{Personas}
\begin{enumerate}
  \item \textbf{Matthew Cruise (Engineering Student)}: Matthew is a 20-year-old second-year Mechanical Engineering student at McMaster University.Hewho lives in a shared house near campus with two close friends. He enjoys attending MES events like pub nights and the Fireball Formal as a way to balance his heavy academic workload with some enjoyment and entertainment. He usually hears about events through word-of-mouth or social media and would prefer an easy and effortless way to find and register for events. Matthew is generally tech savvy and is comfortable using online platforms but doesn't want to be bothered by too many notifications and forms. He has many ideas on how he can generally improve his university experience but doesn't believe he has an outlet to convey them. For Matthew his priority is convenience as he wants to enjoy himself but doesn't want to spend too much time or effort doing so.
  \item \textbf{Adam Clooney (MES Executive)}: Adam Clooney is a 22-year-old final-year Civil Engineering student who currently serves as VP Social on the MES. He is responsible for coordinating large-scale events like Fireball Formal, working closely with other council members to handle logistics, advertising, and student engagement. Adam is outgoing and enjoys bringing people together, but often feels the strain of balancing his role with academic responsibilities. He is proficient with common digital tools such as Google Drive, spreadsheets, and design platforms for promotions, but he’s not highly technical. Adam appreciates structure and tools that keep things organized because he dislikes wasting time fixing errors or repeating work. He is motivated by the sense of accomplishment that comes from hosting a successful event and wants tools that help him stay on top of details.
  \item \textbf{Margot Watson (McMaster Student)}: Margot Watson is a 21-year-old undergraduate student in Political Science at McMaster University. Although she is not part of the engineering faculty, she often attends large MES-hosted events such as the Fireball Formal and other socials because many of her friends are in engineering. Margot lives in an off-campus apartment with two roommates and enjoys being involved in student life across faculties. She has a relaxed attitude toward technology since she uses her phone daily for social media and messaging but she prefers things to be straightforward and intuitive. With money, she is cautious, balancing tuition and living expenses, but she’s willing to spend on experiences with friends. She is motivated by spending time with her friends and having good experiences to have a fullfiling student life.
\end{enumerate}
\subsubsection{Priorities Assigned to Users}
\lips
\subsubsection{User Participation}
\lips
\subsubsection{Maintenance Users and Service Technicians}
\lips

\section{Project Constraints}
\subsection{Mandated Constraints}
\subsubsection{Solution Constraints}
\lips
\subsubsection{Implementation Environment of the Current System}
\lips
\subsubsection{Partner or Collaborative Applications}
\lips
\subsubsection{Off-the-Shelf Software}
\lips
\subsubsection{Anticipated Workplace Environment}
\lips
\subsubsection{Schedule Constraints}
\lips
\subsubsection{Budget Constraints}
\lips
\subsubsection{Enterprise Constraints}
\lips

\subsection{Naming Conventions and Terminology}
\begin{itemize}
  \item \textbf{MES}: McMaster Engineering Society
\end{itemize}
\subsubsection{Glossary of All Terms, Including Acronyms, Used by Stakeholders
involved in the Project}
\lips

\subsection{Relevant Facts And Assumptions}
\subsubsection{Relevant Facts}
\lips
\subsubsection{Business Rules}
\lips
\subsubsection{Assumptions}
\lips

\section{Functional Requirements}
\subsection{The Scope of the Work}
\subsubsection{The Current Situation}
\lips
\subsubsection{The Context of the Work}
\lips
\subsubsection{Work Partitioning}
\lips
\subsubsection{Specifying a Business Use Case (BUC)}
\lips

\subsection{Business Data Model and Data Dictionary}
\subsubsection{Business Data Model}
\lips
\subsubsection{Data Dictionary}
\lips

\subsection{The Scope of the Product}
\subsubsection{Product Boundary}
\lips
\subsubsection{Product Use Case Table}
\lips
\subsubsection{Individual Product Use Cases (PUC's)}
\lips

\subsection{Functional Requirements}
\subsubsection{Functional Requirements}
\lips

\section{Non-Functional Requirements}
\subsection{Look and Feel Requirements}
\subsubsection{Appearance Requirements}
\lips
\subsubsection{Style Requirements}
\lips

\subsection{Usability and Humanity Requirements}
\subsubsection{Ease of Use Requirements}
\lips
\subsubsection{Personalization and Internationalization Requirements}
\lips
\subsubsection{Learning Requirements}
\lips
\subsubsection{Understandability and Politeness Requirements}
\lips
\subsubsection{Accessibility Requirements}
\lips

\subsection{Performance Requirements}
\subsubsection{Speed and Latency Requirements}
\lips
\subsubsection{Safety-Critical Requirements}
\lips
\subsubsection{Precision or Accuracy Requirements}
\lips
\subsubsection{Robustness or Fault-Tolerance Requirements}
\lips
\subsubsection{Capacity Requirements}
\lips
\subsubsection{Scalability or Extensibility Requirements}
\lips
\subsubsection{Longevity Requirements}
\lips

\subsection{Operational and Environmental Requirements}
\subsubsection{Expected Physical Environment}
\lips
\subsubsection{Wider Environment Requirements}
\lips
\subsubsection{Requirements for Interfacing with Adjacent Systems}
\lips
\subsubsection{Productization Requirements}
\lips
\subsubsection{Release Requirements}
\lips

\subsection{Maintainability and Support Requirements}
\subsubsection{Maintenance Requirements}
\lips
\subsubsection{Supportability Requirements}
\lips
\subsubsection{Adaptability Requirements}
\lips

\subsection{Security Requirements}
\subsubsection{Access Requirements}
\lips
\subsubsection{Integrity Requirements}
\lips
\subsubsection{Privacy Requirements}
\lips
\subsubsection{Audit Requirements}
\lips
\subsubsection{Immunity Requirements}
\lips

\subsection{Cultural Requirements}
\subsubsection{Cultural Requirements}
\lips

\subsection{Compliance Requirements}
\subsubsection{Legal Requirements}
\lips
\subsubsection{Standards Compliance Requirements}
\lips

\section{Project Issues}
\subsection{Open Issues}
\lips

\subsection{Off-the-Shelf Solutions}
\subsubsection{Ready-Made Products}
\lips
\subsubsection{Reusable Components}
\lips
\subsubsection{Products That Can Be Copied}
\lips

\subsection{New Problems}
\subsubsection{Effects on the Current Environment}
\lips
\subsubsection{Effects on the Installed Systems}
\lips
\subsubsection{Potential User Problems}
\lips
\subsubsection{Limitations in the Anticipated Implementation Environment That May
Inhibit the New Product}
\lips
\subsubsection{Follow-Up Problems}
\lips

\subsection{Tasks}
\subsubsection{Project Planning}
\lips
\subsubsection{Planning of the Development Phases}
\lips

\subsection{Migration to the New Product}
\subsubsection{Requirements for Migration to the New Product}
\lips
\subsubsection{Data That Has to be Modified or Translated for the New System}
\lips

\subsection{Costs}
\lips
\subsection{User Documentation and Training}
\subsubsection{User Documentation Requirements}
\lips
\subsubsection{Training Requirements}
\lips

\subsection{Waiting Room}
\lips

\subsection{Ideas for Solution}
\lips

\newpage{}
\subsection*{Appendix --- Reflection}

The purpose of reflection questions is to give you a chance to assess your own
learning and that of your group as a whole, and to find ways to improve in the
future. Reflection is an important part of the learning process.  Reflection is
also an essential component of a successful software development process.  

Reflections are most interesting and useful when they're honest, even if the
stories they tell are imperfect. You will be marked based on your depth of
thought and analysis, and not based on the content of the reflections
themselves. Thus, for full marks we encourage you to answer openly and honestly
and to avoid simply writing ``what you think the evaluator wants to hear.''

Please answer the following questions.  Some questions can be answered on the
team level, but where appropriate, each team member should write their own
response:



\subsubsection*{Group Reflection}
\begin{enumerate}
  \item \textbf{How many of your requirements were inspired by speaking to your client(s) or their proxies (e.g. your peers, stakeholders, potential users)?}
  Most of our functional requirements were obtained directly through conversations with out supervisor (Luke) or through todcumentation provided by him. This provided us with a good starting point on what his vision of the system would look like and from there we built upon and added requiremnts based on what we throught the system needed.

  \item \textbf{Which of the courses you have taken, or are currently taking, will help your team to be successful with your capstone project.}
  \begin{itemize}
      \item SFWRENG 4HC3: This course will help us design the frontend components and create a platform which will provide a positive user experience.
      \item SFWRENG 3DB3: Help us design effective database schema and queries.
      \item SFWRENG 3A04: This will help us design our architecture for the system. Also gave us experience working with git control and designing sytem with diagrams.
      \item SFWRENG 2AA4: Designing full scale software and using github project management.
      \item ENG 3PX3 + 2PX3: This course gave us some basic project and team management skills as we had to work on semester long projects similar to what is being done in this course.
      \item SFWRENG 3RA3: Gave us help writing this document and will help us refine and update our requirements in the future.
      \item SFWRENG 3S03: Develop test cases to ensure that our code is working as intended and to allow for CI since we can ensure subsequent changes don't break the whole system.
  \end{itemize}

  \item \textbf{What knowledge and skills will the team collectively need to acquire to successfully complete this capstone project?  Examples of possible knowledge to acquire include domain specific knowledge from the domain of your application, or software engineering knowledge, mechatronics knowledge or computer science knowledge.  Skills may be related to technology, or writing, or presentation, or team management, etc. You should look to identify at least one item for each team member.} \\
  \begin{enumerate}
    \item Full-Stack Web Development: Some skills in this category that will be needed to be learned include requiring the development of front-end and UI. We also need to to develop apis with the restful architecture to connect back-end with our front-end framework. As seen in our document our tech stack includes Vite, React and TanStack Router. These are new technologies to use which we will have to learn.
    \item Software QA and Testing: Using CI/CD to ensure that our code meets the standards that we set and doesn't break when pushing to production. Develop Test Cases to ensure that all code is working as intended and with CI ensure that it automatically is run on each iteration we push to the repository.
    \item Frontend + UI Design: We need to design good user interfaces to ensure that our application will be intuitive to use and will provide users with a positive experience,
    \item Database Management and Design + Data analytics: Create database schema and sql queries to generate and visualize basic data analytics and dashboards.
    \item Software Architecture and Design: Develop an architecture that allows for expandability past the term of our project. We need to acquire skills on different software designs which will allow for better scalability in the future.
    
  \end{enumerate}
  \item \textbf{For each of the knowledge areas and skills identified in the previous question, what are at least two approaches to acquiring the knowledge or mastering the skill?  Of the identified approaches, which will each team member pursue, and why did they make this choice?} \\
  \begin{enumerate}
    \item To acquire this skill we can look to look at the documentation for the technologies. Vite and TanStack have well written documentation on how the framework works as well as how to integrate these frameworks with other technologies we will use in the project. There are also many tutorials and example full-stack projects found on github and accross the internet. Looking at these projects we could gain to understand how to properly implement these technologies and best practices for coding with these frameworks. \\
    \\
    \textbf{Virochaan} and \textbf{Omar} will be looking to improve their skills and knowledge of this domain. Virochaan has past experience working in full-stack development in his co-op so he will be looking to build upon that by using the approaches mentioned above. Omar has been interested in Full-Stack development and has worked on personal projects in this field. He will be looking to transfer the skills he has learnt and use them with these new technologies.\\
    \item To acquire this knowledge we can look back on past courses we did and utilize the testing techniques and methods used in those courses. We can also practice by conducting group peer review sessions of our code and implementing test cases on previous projects and assignments we may have done. \\
    \\
    \textbf{Rayyan} and \textbf{Mohammad} will be looking to improve their skills and knowledge in this domain. Both Rayyan and Mohammad have skills in ths area through past co-op experiences and excelled in the testing course in 3rd year. Mohammad is looking to improve his skills in this category as he believes that this is a useful skill to have for he future career and something he wants to master.
    \item To acquire these skills we can review front-end design principles from our Human-Computer Interaction course and research UI/UX design guidelines. We can also use design tools such as \textit{Figma} to create low-fidelity and high-fidelity mockups of our interfaces before implementing them in code. \\
    \\
    \textbf{Virochaan} and \textbf{Ibrahim} will be looking to improve their skills and knowledge in this domain. Virochaan is interested in improving his skills in this area because he feels that he is weak in this field and finds that having these skills will make him more well-rounded developer. Ibrahim is interested in both full-stack and front-end design and finds that this could be a good experience to gain more exposure in this area.  
    \item  We can use the PostgreSQL documentation and online tutorials to deepen our understanding of how to properly utilize this technology. We can also watch videos on general relational database management as well as effective indexing and querying to make our databases as efficient as possible. 
    \\
    \textbf{Rayyan} and \textbf{Ibrahim} will be looking to improve their skills and knowledge in this domain. Rayyan has experience working on this field in his co-op and found it to be a fulfilling experience. He wants to build upon that foundation and gain more skills working on databases to make him a better back-end developer. Ibrahim perfomed well in the databases course in 3rd year and believes that this could be a good way to build upon that. He always wanted to work on the back-end and database for this project and believes that gaining skills in this area will make him more prepared to accomplish this.\\
    \item To acquire this skill we can study various architectural patterns such as Model-View-Controller (MVC) and Layered Architecture, which are suitable for scalable web systems. We will review course materials from SFWRENG 3A04 and examine open-source project repositories to understand how modular designs are structured. We can also use diagramming tools to model the architecture of our own system before implementation to familiarize ourselves with them. \\
    \\
    \textbf{Omar} and \textbf{Mohammad} will be looking to improve their skills and knowledge in this domain. Omar wishes to pursue this since he wishes to build upon skills he learned in 3A04. In his co-op he had experience working with the MVC architecture and is looking to develop his skills in this area to learn about more patterns. Mohammad also had experience working various architecture types at his co-op. He is looking to put to practice what he learned there in this project while building upon his skillset.
  \end{enumerate}
\end{enumerate}

\subsubsection*{Virochaan Ravichandran Gowri Reflection}
\begin{enumerate}
  \item \textbf{What went well while writing this deliverable?} \\
  During the writing of this deliverable we all had a clear understanding of the goals of this project and the general
  plan which we did in the past deliverable. From this through conversations with our supervisors we were easily able to
  obtain requirements that were relevant to our project. The best part was the communication between the members of our
  team and with the supervisor. This allowed us to
  \item \textbf{What pain points did you experience during this deliverable, and how did you resolve them?} \\
  We needed to make sure that we remained flexible in our requirements if we had to make changes in the future. This was
  especially crucial because we knew that there could be a chance that there could be additional requirements or changes
  based on what the client wanted. Also the client wanted a certain stack so we didn't really have flexibility regarding
  that so it was some new technologies that we need to become accustomed with.

\end{enumerate}

\subsubsection*{Mohammad Mahdi Mahboob Reflection}
\begin{enumerate}
  \item \textbf{What went well while writing this deliverable?} \\
  The projects constraints section write-up went smoothly. Due to previous meetings with the client as well as
  documentation outlined in previous deliverables, much of the constraints were readily available and thereby easily
  articulable. Most constraints regarding the environment of the solution as well as specific requests were
  well-documented in our communications with the client; their technical aptitude was greatly appreciated since they
  were able to provide us with a clear technology and development roadmap to facilitate collaboration across Capstone
  groups for the integrated application.
  \item \textbf{What pain points did you experience during this deliverable, and how did you resolve them?} \\
  Some pain points I encountered during this deliverable were encountered when articulating the requirements. It was
  difficult to gauge what an appropriate level of granularity would be for some of the requirements, as well as decide
  whether requirements for some subsections were necessary. Defining fit criteria was also difficult at times,
  especially for requirements which related to the internal behaviour of the system, such as FR-11. In order to resolve
  these discrepancies, I consulted my teammates and brought forth these concerns during supervisor meetings to
  clarify details, as well as explicitly ascertain the relevance of different requirement categories, such as
  compliance. I would like to commend my teammates in this regard, as well as thank our supervisor for their time and
  expertise.
\end{enumerate}
\subsubsection*{Rayyan Suhail Reflection:}
\begin{enumerate}
  \item \textbf{What went well while writing this deliverable?} \\
  I worked on the Purpose, Relevant Facts and Assumptions, New Problems, Tasks, Migration, and User Documentation sections. What went well was how everything started connecting once we finalized our project direction. Writing the User Documentation section was interesting since we focused on in-app guidance instead of long manuals. The planning sections also came together nicely after a few team discussions, and overall, our communication made the process much smoother.

  \item \textbf{What pain points did you experience during this deliverable, and how did you resolve them?} \\
  The main challenge was keeping the document consistent and not too repetitive across similar sections. Some template areas, like Migration and New Problems, were tricky to adapt to our system at first. I resolved this by reviewing past SRS examples and working closely with my teammates to make sure our sections aligned and stayed concise.
\end{enumerate}







\end{document}