\documentclass[12pt, titlepage]{article}

\usepackage{amsmath, mathtools}

\usepackage[round]{natbib}
\usepackage{amsfonts}
\usepackage{amssymb}
\usepackage{graphicx}
\usepackage{tabularx}
\usepackage{longtable}
\usepackage{colortbl}
\usepackage{xr}
\usepackage{hyperref}
\usepackage{xfrac}
\usepackage{float}
\usepackage{siunitx}
\usepackage{booktabs}
\usepackage{multirow}
\usepackage[section]{placeins}
\usepackage{caption}
\usepackage{fullpage}
\usepackage{float}

\hypersetup{
bookmarks=true,     % show bookmarks bar?
colorlinks=true,       % false: boxed links; true: colored links
linkcolor=red,          % color of internal links (change box color with linkbordercolor)
citecolor=blue,      % color of links to bibliography
filecolor=magenta,  % color of file links
urlcolor=cyan          % color of external links
}

\usepackage{array}

\externaldocument[]{../SoftArchitecture/MG}
\externaldocument[SRS-]{../../SRS/SRS}

\input{../../Comments}
%% Common Parts

\newcommand{\progname}{Software Engineering} % PUT YOUR PROGRAM NAME HERE
\newcommand{\authname}{Team 4, EventHub
\\ Virochaan Ravichandran Gowri
\\ Omar Al-Asfar
\\ Rayyan Suhail
\\ Ibrahim Quraishi
\\ Mohammad Mahdi Mahboob} % AUTHOR NAMES                  

\usepackage{hyperref}
    \hypersetup{colorlinks=true, linkcolor=blue, citecolor=blue, filecolor=blue,
                urlcolor=blue, unicode=false}
    \urlstyle{same}
                                


\newcounter{mnum}
\newcommand{\mthemnum}{M\themnum}
\newcommand{\mref}[1]{M\ref{#1}}

\begin{document}

\title{Module Interface Specification for \progname{}}

\author{\authname}

\date{\today}

\maketitle

\pagenumbering{roman}

\section{Revision History}

\begin{tabularx}{\textwidth}{p{3cm}p{2cm}X}
\toprule {\bf Date} & {\bf Version} & {\bf Notes}\\
\midrule
Date 1 & 1.0 & Notes\\
Date 2 & 1.1 & Notes\\
\bottomrule
\end{tabularx}

~\newpage

\section{Symbols, Abbreviations and Acronyms}

See \hyperref[SRS:glossary]{SRS Glossary}

\wss{Also add any additional symbols, abbreviations or acronyms}

\newpage

\tableofcontents

\newpage

\pagenumbering{arabic}

\section{Introduction}

The following document details the Module Interface Specifications for EvENGage. 
EvENGage is a custom event and survey management system being designed for the MES to simplify and centralize the process of hosting events, conferences, and surveys.

Complementary documents include the System Requirement Specifications
and Module Guide. The full documentation and implementation can be
found at \url{https://github.com/VirochaanRG/MES-Event-Management-System/}.

\section{Notation}

The structure of the MIS for modules comes from \citet{HoffmanAndStrooper1995},
with the addition that template modules have been adapted from
\cite{GhezziEtAl2003}.  The mathematical notation comes from Chapter 3 of
\citet{HoffmanAndStrooper1995}.  For instance, the symbol := is used for a
multiple assignment statement and conditional rules follow the form $(c_1
\Rightarrow r_1 | c_2 \Rightarrow r_2 | ... | c_n \Rightarrow r_n )$.

The following table summarizes the primitive data types used by \progname. 

\begin{center}
\renewcommand{\arraystretch}{1.2}
\noindent 
\begin{tabular}{l l p{7.5cm}} 
\toprule 
\textbf{Data Type} & \textbf{Notation} & \textbf{Description}\\ 
\midrule
character & char & a single symbol or digit \\
integer & $\mathbb{Z}$ & a number without a fractional component in (-$\infty$, $\infty$) \\
natural number & $\mathbb{N}$ & a number without a fractional component in [1, $\infty$) \\
real & $\mathbb{R}$ & any number in (-$\infty$, $\infty$) \\
string & string & An ordered list of characters of any length \\
Cookie & Cookie & A file stored on the client device storing user data \\
List of type T & list(T) & A dynamically sized list of elements of type T \\
Set of type T & set(T) & A dynamically sized set of elements of type T \\
Map of type K to V & map(K, V) & A collection mapping keys of type K to values of type V \\
Date and time & DateTime & A specific date and time using the Gregorian calendar and 24-hour clock \\
User session & Session & Stores data on a user's session and time of last validation \\
\bottomrule
\end{tabular}
\end{center}


\noindent
The specification of \progname \ uses some derived data types: sequences, strings, and
tuples. Sequences are lists filled with elements of the same data type. Strings
are sequences of characters. Tuples contain a list of values, potentially of
different types. In addition, \progname \ uses functions, which
are defined by the data types of their inputs and outputs. Local functions are
described by giving their type signature followed by their specification.

\section{Module Decomposition}

The following table is taken directly from the Module Guide document for this project.

\begin{table}[H]
\centering
\begin{tabular}{p{0.3\textwidth} p{0.6\textwidth}}
\toprule
\textbf{Level 1} & \textbf{Level 2}\\
\midrule
\multirow{9}{0.3\textwidth}{Behaviour-Hiding Modules} 
& M\ref{m1}: User Authentication Module\\
& M\ref{m2}: User Authorization Module\\
& M\ref{m3}: Form Template Module\\
& M\ref{m4}: Form Submission Module\\
& M\ref{m5}: Event Management Module\\
& M\ref{m6}: Event Notification Module\\
& M\ref{m7}: Registration Module\\
& M\ref{m8}: Attendance Tracking Module\\
& M\ref{m9}: Report Generation Module\\
\midrule

\multirow{3}{0.3\textwidth}{Software-Decision Modules} 
& M\ref{m10}: Analytics Module\\
& M\ref{m11}: Database Access Module\\
& M\ref{m12}: Audit Module\\
\bottomrule

\end{tabular}
\caption{Module Hierarchy}
\label{TblMH}
\end{table}

\newpage

\section{MIS of M\ref{m1}: User Authentication Module}

\subsection{Module}

Contains functionality for logging in and authenticating users.

\subsection{Uses}

\subsection{Syntax}

\subsubsection{Exported Constants}

None

\subsubsection{Exported Access Programs}

\begin{center}
\begin{tabularx}{\textwidth}{p{0.2\textwidth} p{0.3\textwidth} p{0.3\textwidth} p{0.2\textwidth}}
\hline
\textbf{Name} & \textbf{In} & \textbf{Out} & \textbf{Exceptions} \\
\hline
\texttt{attemptLogin} & \texttt{username : string } \newline \texttt{password : string} & \texttt{sessionCookie : Cookie} & None \\
\hline
\texttt{logout} & \texttt{sessionCookie : Cookie }  & None & \texttt{InvalidSession} \\
\hline
\texttt{validateSession} & \texttt{sessionCookie : Cookie} & \texttt{sessionIsValid : bool} & None\\
\hline
\texttt{registerUser} & \texttt{username : string } \newline \texttt{password : string} & None & None \\
\hline
\end{tabularx}
\end{center}

\subsection{Semantics}

\subsubsection{State Variables}

\texttt{sessions : set(Session)}: Set of all active sessions

\subsubsection{Environment Variables}

\texttt{currentTime : DateTime}: Stores the current date and time

\subsubsection{Assumptions}

None

\subsubsection{Access Routine Semantics}

\noindent \texttt{attemptLogin(username : string, password : string)}: \\
Attempts a login given a username and password.
\begin{itemize}
\item transition: creates and adds a new session to \texttt{sessions} if login is successful
\item output: Cookie containing session data to be sent back to the client
\item exception: None 
\end{itemize}

\noindent \texttt{logout(sessionCookie: Cookie)}: \\
Logs the specified user out.
\begin{itemize}
\item transition: Removes the specified session from \texttt{sessions}
\item output: None
\item exception: InvalidSession if session is not found in \texttt{sessions} 
\end{itemize}

\noindent \texttt{validateSession(sessionCookie: Cookie)}: \\
Validates and refreshes a users session.
\begin{itemize}
\item transition: Refreshes the timeout of the session in \texttt{sessions} if session is valid
\item output: Boolean indicating if sessionCookie exists in \texttt{sessions}
\item exception: None
\end{itemize}

\noindent \texttt{registerUser(username : string, password : string)}: \\
Registers a new user to the database of users.
\begin{itemize}
\item transition: None
\item output: Boolean stating whether registration was successful
\item exception: None
\end{itemize}

\subsubsection{Local Functions}

\noindent \texttt{refreshSessions()}:
\begin{itemize}
\item transition: Periodically checks \texttt{sessions} and removes any session older than 3 hours
\item output: None
\item exception: None 
\end{itemize}

\newpage

\section{MIS of M\ref{MG-m2}: User Authorization Module}

\subsection{Module}

Contains functionality for logging in and authenticating

\subsection{Uses}


\subsection{Syntax}

\subsubsection{Exported Constants}

\subsubsection{Exported Access Programs}

\begin{center}
\begin{tabular}{p{2cm} p{4cm} p{4cm} p{2cm}}
\hline
\textbf{Name} & \textbf{In} & \textbf{Out} & \textbf{Exceptions} \\
\hline
\wss{accessProg} & - & - & - \\
\hline
\end{tabular}
\end{center}

\subsection{Semantics}

\subsubsection{State Variables}

\wss{Not all modules will have state variables.  State variables give the module
  a memory.}

\subsubsection{Environment Variables}

\wss{This section is not necessary for all modules.  Its purpose is to capture
  when the module has external interaction with the environment, such as for a
  device driver, screen interface, keyboard, file, etc.}

\subsubsection{Assumptions}

\wss{Try to minimize assumptions and anticipate programmer errors via
  exceptions, but for practical purposes assumptions are sometimes appropriate.}

\subsubsection{Access Routine Semantics}

\noindent \wss{accessProg}():
\begin{itemize}
\item transition: \wss{if appropriate} 
\item output: \wss{if appropriate} 
\item exception: \wss{if appropriate} 
\end{itemize}

\wss{A module without environment variables or state variables is unlikely to
  have a state transition.  In this case a state transition can only occur if
  the module is changing the state of another module.}

\wss{Modules rarely have both a transition and an output.  In most cases you
  will have one or the other.}

\subsubsection{Local Functions}

\wss{As appropriate} \wss{These functions are for the purpose of specification.
  They are not necessarily something that is going to be implemented
  explicitly.  Even if they are implemented, they are not exported; they only
  have local scope.}

\newpage

\section{MIS of M\ref{m3}: Form Template Module}

\subsection{Module}

Contains functionality for logging in and authenticating

\subsection{Uses}


\subsection{Syntax}

\subsubsection{Exported Constants}

\subsubsection{Exported Access Programs}

\begin{center}
\begin{tabular}{p{2cm} p{4cm} p{4cm} p{2cm}}
\hline
\textbf{Name} & \textbf{In} & \textbf{Out} & \textbf{Exceptions} \\
\hline
\wss{accessProg} & - & - & - \\
\hline
\end{tabular}
\end{center}

\subsection{Semantics}

\subsubsection{State Variables}

\wss{Not all modules will have state variables.  State variables give the module
  a memory.}

\subsubsection{Environment Variables}

\wss{This section is not necessary for all modules.  Its purpose is to capture
  when the module has external interaction with the environment, such as for a
  device driver, screen interface, keyboard, file, etc.}

\subsubsection{Assumptions}

\wss{Try to minimize assumptions and anticipate programmer errors via
  exceptions, but for practical purposes assumptions are sometimes appropriate.}

\subsubsection{Access Routine Semantics}

\noindent \wss{accessProg}():
\begin{itemize}
\item transition: \wss{if appropriate} 
\item output: \wss{if appropriate} 
\item exception: \wss{if appropriate} 
\end{itemize}

\wss{A module without environment variables or state variables is unlikely to
  have a state transition.  In this case a state transition can only occur if
  the module is changing the state of another module.}

\wss{Modules rarely have both a transition and an output.  In most cases you
  will have one or the other.}

\subsubsection{Local Functions}

\wss{As appropriate} \wss{These functions are for the purpose of specification.
  They are not necessarily something that is going to be implemented
  explicitly.  Even if they are implemented, they are not exported; they only
  have local scope.}

\newpage

\section{MIS of M\ref{m4}: Form Submission Module}

\subsection{Module}

Contains functionality for logging in and authenticating

\subsection{Uses}


\subsection{Syntax}

\subsubsection{Exported Constants}

\subsubsection{Exported Access Programs}

\begin{center}
\begin{tabular}{p{2cm} p{4cm} p{4cm} p{2cm}}
\hline
\textbf{Name} & \textbf{In} & \textbf{Out} & \textbf{Exceptions} \\
\hline
\wss{accessProg} & - & - & - \\
\hline
\end{tabular}
\end{center}

\subsection{Semantics}

\subsubsection{State Variables}

\wss{Not all modules will have state variables.  State variables give the module
  a memory.}

\subsubsection{Environment Variables}

\wss{This section is not necessary for all modules.  Its purpose is to capture
  when the module has external interaction with the environment, such as for a
  device driver, screen interface, keyboard, file, etc.}

\subsubsection{Assumptions}

\wss{Try to minimize assumptions and anticipate programmer errors via
  exceptions, but for practical purposes assumptions are sometimes appropriate.}

\subsubsection{Access Routine Semantics}

\noindent \wss{accessProg}():
\begin{itemize}
\item transition: \wss{if appropriate} 
\item output: \wss{if appropriate} 
\item exception: \wss{if appropriate} 
\end{itemize}

\wss{A module without environment variables or state variables is unlikely to
  have a state transition.  In this case a state transition can only occur if
  the module is changing the state of another module.}

\wss{Modules rarely have both a transition and an output.  In most cases you
  will have one or the other.}

\subsubsection{Local Functions}

\wss{As appropriate} \wss{These functions are for the purpose of specification.
  They are not necessarily something that is going to be implemented
  explicitly.  Even if they are implemented, they are not exported; they only
  have local scope.}

\newpage

\section{MIS of M\ref{m5}: Event Management Module}

\subsection{Module}

Contains functionality for logging in and authenticating

\subsection{Uses}


\subsection{Syntax}

\subsubsection{Exported Constants}

\subsubsection{Exported Access Programs}

\begin{center}
\begin{tabular}{p{2cm} p{4cm} p{4cm} p{2cm}}
\hline
\textbf{Name} & \textbf{In} & \textbf{Out} & \textbf{Exceptions} \\
\hline
\wss{accessProg} & - & - & - \\
\hline
\end{tabular}
\end{center}

\subsection{Semantics}

\subsubsection{State Variables}

\wss{Not all modules will have state variables.  State variables give the module
  a memory.}

\subsubsection{Environment Variables}

\wss{This section is not necessary for all modules.  Its purpose is to capture
  when the module has external interaction with the environment, such as for a
  device driver, screen interface, keyboard, file, etc.}

\subsubsection{Assumptions}

\wss{Try to minimize assumptions and anticipate programmer errors via
  exceptions, but for practical purposes assumptions are sometimes appropriate.}

\subsubsection{Access Routine Semantics}

\noindent \wss{accessProg}():
\begin{itemize}
\item transition: \wss{if appropriate} 
\item output: \wss{if appropriate} 
\item exception: \wss{if appropriate} 
\end{itemize}

\wss{A module without environment variables or state variables is unlikely to
  have a state transition.  In this case a state transition can only occur if
  the module is changing the state of another module.}

\wss{Modules rarely have both a transition and an output.  In most cases you
  will have one or the other.}

\subsubsection{Local Functions}

\wss{As appropriate} \wss{These functions are for the purpose of specification.
  They are not necessarily something that is going to be implemented
  explicitly.  Even if they are implemented, they are not exported; they only
  have local scope.}

\newpage

\section{MIS of M\ref{m6}: Event Notification Module}

\subsection{Module}

Contains functionality for notifying users abouts upcoming events.

\subsection{Uses}

\subsection{Syntax}

\subsubsection{Exported Constants}

None

\subsubsection{Exported Access Programs}

\begin{center}
\begin{tabular}{p{2cm} p{4cm} p{4cm} p{2cm}}
\hline
\textbf{Name} & \textbf{In} & \textbf{Out} & \textbf{Exceptions} \\
\hline
\wss{accessProg} & - & - & - \\
\hline
\end{tabular}
\end{center}

\subsection{Semantics}

\subsubsection{State Variables}

\wss{Not all modules will have state variables.  State variables give the module
  a memory.}

\subsubsection{Environment Variables}

\wss{This section is not necessary for all modules.  Its purpose is to capture
  when the module has external interaction with the environment, such as for a
  device driver, screen interface, keyboard, file, etc.}

\subsubsection{Assumptions}

\wss{Try to minimize assumptions and anticipate programmer errors via
  exceptions, but for practical purposes assumptions are sometimes appropriate.}

\subsubsection{Access Routine Semantics}

\noindent \wss{accessProg}():
\begin{itemize}
\item transition: \wss{if appropriate} 
\item output: \wss{if appropriate} 
\item exception: \wss{if appropriate} 
\end{itemize}

\wss{A module without environment variables or state variables is unlikely to
  have a state transition.  In this case a state transition can only occur if
  the module is changing the state of another module.}

\wss{Modules rarely have both a transition and an output.  In most cases you
  will have one or the other.}

\subsubsection{Local Functions}

\wss{As appropriate} \wss{These functions are for the purpose of specification.
  They are not necessarily something that is going to be implemented
  explicitly.  Even if they are implemented, they are not exported; they only
  have local scope.}

\newpage

\section{MIS of M\ref{m8}: Registration Module}

\subsection{Module}

Contains functionality for registering users for events.

\subsection{Uses}
\begin{itemize}
  \item Event Management Module (M\ref{m7})
\end{itemize}

\subsection{Syntax}

\subsubsection{Exported Constants}

\subsubsection{Exported Access Programs}

\begin{center}
\begin{tabular}{p{2cm} p{4cm} p{4cm} p{2cm}}
\hline
\textbf{Name} & \textbf{In} & \textbf{Out} & \textbf{Exceptions} \\
\hline
\wss{accessProg} & - & - & - \\
\hline
\end{tabular}
\end{center}

\subsection{Semantics}

\subsubsection{State Variables}

\wss{Not all modules will have state variables.  State variables give the module
  a memory.}

\subsubsection{Environment Variables}

\wss{This section is not necessary for all modules.  Its purpose is to capture
  when the module has external interaction with the environment, such as for a
  device driver, screen interface, keyboard, file, etc.}

\subsubsection{Assumptions}

\wss{Try to minimize assumptions and anticipate programmer errors via
  exceptions, but for practical purposes assumptions are sometimes appropriate.}

\subsubsection{Access Routine Semantics}

\noindent \wss{accessProg}():
\begin{itemize}
\item transition: \wss{if appropriate} 
\item output: \wss{if appropriate} 
\item exception: \wss{if appropriate} 
\end{itemize}

\wss{A module without environment variables or state variables is unlikely to
  have a state transition.  In this case a state transition can only occur if
  the module is changing the state of another module.}

\wss{Modules rarely have both a transition and an output.  In most cases you
  will have one or the other.}

\subsubsection{Local Functions}

\wss{As appropriate} \wss{These functions are for the purpose of specification.
  They are not necessarily something that is going to be implemented
  explicitly.  Even if they are implemented, they are not exported; they only
  have local scope.}

\newpage

\section{MIS of M\ref{m8}: Attendance Tracking Module}

\subsection{Module}

Contains functionality for logging in and authenticating

\subsection{Uses}


\subsection{Syntax}

\subsubsection{Exported Constants}

\subsubsection{Exported Access Programs}

\begin{center}
\begin{tabular}{p{2cm} p{4cm} p{4cm} p{2cm}}
\hline
\textbf{Name} & \textbf{In} & \textbf{Out} & \textbf{Exceptions} \\
\hline
\wss{accessProg} & - & - & - \\
\hline
\end{tabular}
\end{center}

\subsection{Semantics}

\subsubsection{State Variables}

\wss{Not all modules will have state variables.  State variables give the module
  a memory.}

\subsubsection{Environment Variables}

\wss{This section is not necessary for all modules.  Its purpose is to capture
  when the module has external interaction with the environment, such as for a
  device driver, screen interface, keyboard, file, etc.}

\subsubsection{Assumptions}

\wss{Try to minimize assumptions and anticipate programmer errors via
  exceptions, but for practical purposes assumptions are sometimes appropriate.}

\subsubsection{Access Routine Semantics}

\noindent \wss{accessProg}():
\begin{itemize}
\item transition: \wss{if appropriate} 
\item output: \wss{if appropriate} 
\item exception: \wss{if appropriate} 
\end{itemize}

\wss{A module without environment variables or state variables is unlikely to
  have a state transition.  In this case a state transition can only occur if
  the module is changing the state of another module.}

\wss{Modules rarely have both a transition and an output.  In most cases you
  will have one or the other.}

\subsubsection{Local Functions}

\wss{As appropriate} \wss{These functions are for the purpose of specification.
  They are not necessarily something that is going to be implemented
  explicitly.  Even if they are implemented, they are not exported; they only
  have local scope.}

\newpage

\section{MIS of M\ref{m9}: Report Generation Module}

\subsection{Module}

Contains functionality for logging in and authenticating

\subsection{Uses}


\subsection{Syntax}

\subsubsection{Exported Constants}

\subsubsection{Exported Access Programs}

\begin{center}
\begin{tabular}{p{2cm} p{4cm} p{4cm} p{2cm}}
\hline
\textbf{Name} & \textbf{In} & \textbf{Out} & \textbf{Exceptions} \\
\hline
\wss{accessProg} & - & - & - \\
\hline
\end{tabular}
\end{center}

\subsection{Semantics}

\subsubsection{State Variables}

\wss{Not all modules will have state variables.  State variables give the module
  a memory.}

\subsubsection{Environment Variables}

\wss{This section is not necessary for all modules.  Its purpose is to capture
  when the module has external interaction with the environment, such as for a
  device driver, screen interface, keyboard, file, etc.}

\subsubsection{Assumptions}

\wss{Try to minimize assumptions and anticipate programmer errors via
  exceptions, but for practical purposes assumptions are sometimes appropriate.}

\subsubsection{Access Routine Semantics}

\noindent \wss{accessProg}():
\begin{itemize}
\item transition: \wss{if appropriate} 
\item output: \wss{if appropriate} 
\item exception: \wss{if appropriate} 
\end{itemize}

\wss{A module without environment variables or state variables is unlikely to
  have a state transition.  In this case a state transition can only occur if
  the module is changing the state of another module.}

\wss{Modules rarely have both a transition and an output.  In most cases you
  will have one or the other.}

\subsubsection{Local Functions}

\wss{As appropriate} \wss{These functions are for the purpose of specification.
  They are not necessarily something that is going to be implemented
  explicitly.  Even if they are implemented, they are not exported; they only
  have local scope.}

\newpage

\section{MIS of M\ref{m10}: Analytics Module}

\subsection{Module}

Contains functionality for computing summary statistics and trends for events,
registrations, attendance, and surveys.

\subsection{Uses}

Database Access Module (M\ref{m11}) \\
Registration Module (M\ref{m7}) \\
Attendance Tracking Module (M\ref{m8}) \\
Survey Response Module (M\ref{m9})

\subsection{Syntax}

\subsubsection{Exported Constants}

None

\subsubsection{Exported Access Programs}

\begin{center}
\begin{tabularx}{\textwidth}{p{0.27\textwidth} p{0.19\textwidth} p{0.29\textwidth} p{0.20\textwidth}}
\hline
\textbf{Name} & \textbf{In} & \textbf{Out} & \textbf{Exceptions} \\
\hline
\texttt{getEventStats} &
\texttt{eventId : $\mathbb{N}$} &
\texttt{stats : EventStats} &
\texttt{EventNotFound} \\
\hline
\texttt{getSurveyStats} &
\texttt{surveyId : $\mathbb{N}$} &
\texttt{summary : SurveySummary} &
\texttt{SurveyNotFound} \\
\hline
\texttt{getRegistrationTrends} &
\texttt{eventId : $\mathbb{N}$} &
\texttt{trend : list(RegistrationPoint)} &
\texttt{EventNotFound} \\
\hline
\end{tabularx}
\end{center}


\subsection{Semantics}

\subsubsection{State Variables}

None

\subsubsection{Environment Variables}

None

\subsubsection{Assumptions}

\begin{itemize}
  \item The Database Access Module (M\ref{m11}) correctly stores registrations,
        attendance records, and survey responses.
  \item The given \texttt{eventId} and \texttt{surveyId} refer to events and
        surveys that were, if they exist, created by other modules.
\end{itemize}

\subsubsection{Access Routine Semantics}

\noindent \texttt{getEventStats(eventId : $\mathbb{N}$)}:
\begin{itemize}
  \item transition: None (read-only). Queries the database for all registrations
        and attendance records associated with \texttt{eventId} and computes
        totals such as number of registrations, check-ins, and attendance rate.
  \item output: \texttt{stats} containing the computed summary values.
  \item exception:
    \begin{itemize}
      \item \texttt{EventNotFound} if \texttt{eventId} does not correspond to any event.
    \end{itemize}
\end{itemize}

\noindent \texttt{getSurveyStats(surveyId : $\mathbb{N}$)}:
\begin{itemize}
  \item transition: None (read-only). Queries survey responses associated with
        \texttt{surveyId} and aggregates them per question (for example,
        counts per option for multiple-choice questions).
  \item output: \texttt{summary} containing aggregated statistics for each question.
  \item exception:
    \begin{itemize}
      \item \texttt{SurveyNotFound} if \texttt{surveyId} does not correspond to any survey.
    \end{itemize}
\end{itemize}

\noindent \texttt{getRegistrationTrends(eventId : $\mathbb{N}$)}:
\begin{itemize}
  \item transition: None (read-only). Retrieves timestamps of registrations for
        \texttt{eventId} and computes cumulative registration counts over time.
  \item output: \texttt{trend} as a list of \texttt{RegistrationPoint} records,
        each containing a timestamp and cumulative registration count.
  \item exception:
    \begin{itemize}
      \item \texttt{EventNotFound} if \texttt{eventId} does not correspond to any event.
    \end{itemize}
\end{itemize}

\subsubsection{Local Functions}

None

\newpage

\section{MIS of M\ref{m11}: Database Access Module}

\subsection{Module}

Provides a generic interface for reading from and writing to the application's
PostgreSQL database. Responsible for connection management, transactions,
and basic CRUD operations used by higher-level modules (e.g., event
management, registration, analytics).

\subsection{Uses}

None

\subsection{Syntax}

\subsubsection{Exported Constants}

None

\subsubsection{Exported Access Programs}

\begin{center}
\begin{tabularx}{\textwidth}{p{0.15\textwidth} p{0.36\textwidth} p{0.20\textwidth} p{0.20\textwidth}}
\hline
\textbf{Name} & \textbf{In} & \textbf{Out} & \textbf{Exceptions} \\
\hline
\texttt{fetchRow} &
\texttt{table : string, id : $\mathbb{N}$} &
\texttt{row : Record} &
\texttt{RecordNotFound, DatabaseError} \\
\hline
\texttt{fetchRows} &
\texttt{table : string, filter : FilterExpr} &
\texttt{rows : list(Record)} &
\texttt{DatabaseError} \\
\hline
\texttt{insertRow} &
\texttt{table : string, data : Record} &
\texttt{id : $\mathbb{N}$} &
\texttt{DatabaseError} \\
\hline
\texttt{updateRow} &
\texttt{table : string, id : $\mathbb{N}$, data : Record} &
\texttt{-} &
\texttt{RecordNotFound, DatabaseError} \\
\hline
\texttt{deleteRow} &
\texttt{table : string, id : $\mathbb{N}$} &
\texttt{-} &
\texttt{RecordNotFound, DatabaseError} \\
\hline

\end{tabularx}
\end{center}


\subsection{Semantics}

\subsubsection{State Variables}

\begin{itemize}
  \item connectionPool : ConnectionPool \\
        Pool of reusable connections to the PostgreSQL database.
  \item activeTx : set(TransactionId) \\
        Set of identifiers for currently active transactions.
\end{itemize}

\subsubsection{Environment Variables}

\begin{itemize}
  \item dbServer : PostgreSQLInstance \\
        Running PostgreSQL database server hosting the application schema.
\end{itemize}

\subsubsection{Assumptions}

\begin{itemize}
  \item The database schema has been created and migrated before any access
        program is invoked.
  \item \texttt{connectionPool} is initialized during system startup.
  \item \texttt{Record} and \texttt{FilterExpr} are abstract data structures whose
        concrete representation is handled by the ORM / query layer.
\end{itemize}

\subsubsection{Access Routine Semantics}

\noindent fetchRow(table : string, id : N):
\begin{itemize}
  \item transition: None.
  \item output: Returns the row in \texttt{table} whose primary key equals \texttt{id}.
  \item exception: 
        \begin{itemize}
          \item RecordNotFound if no matching row exists.
          \item DatabaseError if a low-level database error occurs.
        \end{itemize}
\end{itemize}

\noindent fetchRows(table : string, filter : FilterExpr):
\begin{itemize}
  \item transition: None.
  \item output: Returns all rows in \texttt{table} that satisfy \texttt{filter}.
  \item exception: DatabaseError if a low-level database error occurs.
\end{itemize}

\noindent insertRow(table : string, data : Record):
\begin{itemize}
  \item transition: Inserts a new row into \texttt{table} populated with the fields in \texttt{data}.
  \item output: Returns the primary key \texttt{id} assigned to the new row.
  \item exception: DatabaseError if the insert fails (e.g., constraint violation, connectivity issues).
\end{itemize}

\noindent updateRow(table : string, id : N, data : Record):
\begin{itemize}
  \item transition: Updates the existing row in \texttt{table} with primary key \texttt{id}
        using the fields in \texttt{data}.
  \item output: None.
  \item exception:
        \begin{itemize}
          \item RecordNotFound if no matching row exists.
          \item DatabaseError if the update fails.
        \end{itemize}
\end{itemize}

\noindent deleteRow(table : string, id : N):
\begin{itemize}
  \item transition: Removes the row in \texttt{table} whose primary key equals \texttt{id}.
  \item output: None.
  \item exception:
        \begin{itemize}
          \item RecordNotFound if no matching row exists.
          \item DatabaseError if the delete fails.
        \end{itemize}
\end{itemize}

\subsubsection{Local Functions}

\begin{itemize}
  \item acquireConnection() : Connection \\
        Obtains a connection from \texttt{connectionPool}, opening a new one if required.
  \item releaseConnection(c : Connection) \\
        Returns \texttt{c} to \texttt{connectionPool} or closes it on error.
  \item mapRowToRecord(raw : Row) : Record \\
        Maps a raw database row to the abstract \texttt{Record} structure.
\end{itemize}

\newpage

\section{MIS of M\ref{m12}: Audit Module}

\subsection{Module}

Contains functionality for recording administrative and sensitive system
actions in an append-only audit log for traceability and accountability.

\subsection{Uses}

Database Access Module (M\ref{m11})

\subsection{Syntax}

\subsubsection{Exported Constants}

None

\subsubsection{Exported Access Programs}

\begin{center}
\begin{tabularx}{\textwidth}{p{0.21\textwidth} p{0.27\textwidth} p{0.20\textwidth} p{0.20\textwidth}}
\hline
\textbf{Name} & \textbf{In} & \textbf{Out} & \textbf{Exceptions} \\
\hline
\texttt{recordEvent} &
\texttt{entry : AuditEntry} &
\texttt{-} &
\texttt{DatabaseError} \\
\hline
\texttt{getAuditLog} &
\texttt{filter : AuditFilter} &
\texttt{entries : list(AuditEntry)} &
\texttt{DatabaseError} \\
\hline

\end{tabularx}
\end{center}

\subsection{Semantics}

\subsubsection{State Variables}

\begin{itemize}
  \item auditLog : set(AuditEntry) \\
        Abstract representation of all recorded audit entries.
\end{itemize}

\subsubsection{Environment Variables}

None

\subsubsection{Assumptions}

\begin{itemize}
  \item Audit entries produced by other modules accurately describe the action taken.
  \item The underlying database schema includes an audit log table.
\end{itemize}

\subsubsection{Access Routine Semantics}

\noindent\texttt{recordEvent(entry : AuditEntry)}:
\begin{itemize}
  \item transition: Adds \texttt{entry} to \texttt{auditLog} and persists it through
        the Database Access Module.
  \item output: None
  \item exception:
    \begin{itemize}
      \item \texttt{DatabaseError} if the entry could not be written.
    \end{itemize}
\end{itemize}

\noindent\texttt{getAuditLog(filter : AuditFilter)}:
\begin{itemize}
  \item transition: None (read-only)
  \item output: Returns all \texttt{AuditEntry} values matching the given filter,
        such as by user, action type, or date range.
  \item exception:
    \begin{itemize}
      \item \texttt{DatabaseError} if retrieval fails.
    \end{itemize}
\end{itemize}

\subsubsection{Local Functions}

None

\newpage

\bibliographystyle {plainnat}
\bibliography {../../../refs/References}

\newpage

\section{Appendix} \label{Appendix}

\wss{Extra information if required}

\newpage{}

\section*{Appendix --- Reflection}

\wss{Not required for CAS 741 projects}

The information in this section will be used to evaluate the team members on the
graduate attribute of Problem Analysis and Design.

\input{../../Reflection.tex}

\begin{enumerate}
  \item What went well while writing this deliverable? 
  \item What pain points did you experience during this deliverable, and how
    did you resolve them?
  \item Which of your design decisions stemmed from speaking to your client(s)
  or a proxy (e.g. your peers, stakeholders, potential users)? For those that
  were not, why, and where did they come from?
  \item While creating the design doc, what parts of your other documents (e.g.
  requirements, hazard analysis, etc), it any, needed to be changed, and why?
  \item What are the limitations of your solution?  Put another way, given
  unlimited resources, what could you do to make the project better? (LO\_ProbSolutions)
  \item Give a brief overview of other design solutions you considered.  What
  are the benefits and tradeoffs of those other designs compared with the chosen
  design?  From all the potential options, why did you select the documented design?
  (LO\_Explores)
\end{enumerate}


\end{document}