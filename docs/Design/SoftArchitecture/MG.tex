\documentclass[12pt, titlepage]{article}

\usepackage{fullpage}
\usepackage[round]{natbib}
\usepackage{multirow}
\usepackage{booktabs}
\usepackage{tabularx}
\usepackage{graphicx}
\usepackage{float}
\usepackage{hyperref}
\hypersetup{
    colorlinks,
    citecolor=blue,
    filecolor=black,
    linkcolor=red,
    urlcolor=blue
}

\input{../../Comments}
%% Common Parts

\newcommand{\progname}{Software Engineering} % PUT YOUR PROGRAM NAME HERE
\newcommand{\authname}{Team 4, EventHub
\\ Virochaan Ravichandran Gowri
\\ Omar Al-Asfar
\\ Rayyan Suhail
\\ Ibrahim Quraishi
\\ Mohammad Mahdi Mahboob} % AUTHOR NAMES                  

\usepackage{hyperref}
    \hypersetup{colorlinks=true, linkcolor=blue, citecolor=blue, filecolor=blue,
                urlcolor=blue, unicode=false}
    \urlstyle{same}
                                


\newcounter{acnum}
\newcommand{\actheacnum}{AC\theacnum}
\newcommand{\acref}[1]{AC\ref{#1}}

\newcounter{ucnum}
\newcommand{\uctheucnum}{UC\theucnum}
\newcommand{\uref}[1]{UC\ref{#1}}

\newcounter{mnum}
\newcommand{\mthemnum}{M\themnum}
\newcommand{\mref}[1]{M\ref{#1}}

\begin{document}

\title{Module Guide for \progname{}} 
\author{\authname}
\date{\today}

\maketitle

\pagenumbering{roman}

\section{Revision History}

\begin{tabularx}{\textwidth}{p{3cm}p{2cm}X}
\toprule {\bf Date} & {\bf Version} & {\bf Notes}\\
\midrule
Date 1 & 1.0 & Notes\\
Date 2 & 1.1 & Notes\\
\bottomrule
\end{tabularx}

\newpage

\section{Reference Material}

This section records information for easy reference.

\subsection{Abbreviations and Acronyms}

\renewcommand{\arraystretch}{1.2}
\begin{tabular}{l l} 
  \toprule		
  \textbf{symbol} & \textbf{description}\\
  \midrule 
  AC & Anticipated Change\\
  DAG & Directed Acyclic Graph \\
  M & Module \\
  MG & Module Guide \\
  OS & Operating System \\
  R & Requirement\\
  SC & Scientific Computing \\
  SRS & Software Requirements Specification\\
  \progname & Explanation of program name\\
  UC & Unlikely Change \\
  \wss{etc.} & \wss{...}\\
  \bottomrule
\end{tabular}\\

\newpage

\tableofcontents

\listoftables

\listoffigures

\newpage

\pagenumbering{arabic}

\section{Introduction}

Decomposing a system into modules is a commonly accepted approach to developing
software.  A module is a work assignment for a programmer or programming
team~\citep{ParnasEtAl1984}.  We advocate a decomposition
based on the principle of information hiding~\citep{Parnas1972a}.  This
principle supports design for change, because the ``secrets'' that each module
hides represent likely future changes.  Design for change is valuable in SC,
where modifications are frequent, especially during initial development as the
solution space is explored.  

Our design follows the rules layed out by \citet{ParnasEtAl1984}, as follows:
\begin{itemize}
\item System details that are likely to change independently should be the
  secrets of separate modules.
\item Each data structure is implemented in only one module.
\item Any other program that requires information stored in a module's data
  structures must obtain it by calling access programs belonging to that module.
\end{itemize}

After completing the first stage of the design, the Software Requirements
Specification (SRS), the Module Guide (MG) is developed~\citep{ParnasEtAl1984}. The MG
specifies the modular structure of the system and is intended to allow both
designers and maintainers to easily identify the parts of the software.  The
potential readers of this document are as follows:

\begin{itemize}
\item New project members: This document can be a guide for a new project member
  to easily understand the overall structure and quickly find the
  relevant modules they are searching for.
\item Maintainers: The hierarchical structure of the module guide improves the
  maintainers' understanding when they need to make changes to the system. It is
  important for a maintainer to update the relevant sections of the document
  after changes have been made.
\item Designers: Once the module guide has been written, it can be used to
  check for consistency, feasibility, and flexibility. Designers can verify the
  system in various ways, such as consistency among modules, feasibility of the
  decomposition, and flexibility of the design.
\end{itemize}

The rest of the document is organized as follows. Section
\ref{SecChange} lists the anticipated and unlikely changes of the software
requirements. Section \ref{SecMH} summarizes the module decomposition that
was constructed according to the likely changes. Section \ref{SecConnection}
specifies the connections between the software requirements and the
modules. Section \ref{SecMD} gives a detailed description of the
modules. Section \ref{SecTM} includes two traceability matrices. One checks
the completeness of the design against the requirements provided in the SRS. The
other shows the relation between anticipated changes and the modules. Section
\ref{SecUse} describes the use relation between modules.

\section{Anticipated and Unlikely Changes} \label{SecChange}

This section lists possible changes to the system. According to the likeliness
of the change, the possible changes are classified into two
categories. Anticipated changes are listed in Section \ref{SecAchange}, and
unlikely changes are listed in Section \ref{SecUchange}.

\subsection{Anticipated Changes} \label{SecAchange}

Anticipated changes are the source of the information that is to be hidden
inside the modules. Ideally, changing one of the anticipated changes will only
require changing the one module that hides the associated decision. The approach
adapted here is called design for change.

\begin{description}
\item[\refstepcounter{acnum} \actheacnum \label{acUI}:] 
  The user interface of the app. Modifications may be made to improve usability, accessibility, or better represent the use case (eg. MES).
\item[\refstepcounter{acnum} \actheacnum \label{acForms}:] 
  The form builder functionality. The structure and logic of registration and feedback forms may change to support new fields, features, or requirements.
\item[\refstepcounter{acnum} \actheacnum \label{acAuth}:] 
  The user authentication and authorization policy. The security and user requirements may evolve, requiring the introduction of new authentication techniques (eg. custom login, university single sign-on). There may also be a need to add new user/admin roles, or modify the permissions process based on the use case.
\item[\refstepcounter{acnum} \actheacnum \label{acData}:] 
  Backend functionality. The database handling or entry relationships may change to support scalability, or custom event workflows.
\item[\refstepcounter{acnum} \actheacnum \label{acNotifications}:] 
  Notification delivery methods. Communication channels (email, SMS, push notifications) and message formats may improve to meet updated communication needs for event reminders, updates, or confirmations.
\item[\refstepcounter{acnum} \actheacnum \label{acSecurity}:] 
  Security and encryption network. The app's security standards may require updated encryption algorithms, expanded input validation, and refined logging detail depending on the use case to ensure compliance.
\item[\refstepcounter{acnum} \actheacnum \label{acIntegration}:] 
  External integration methods. The app may be required to connect with specialized APIs, external databases, or other third-party systems (eg. payment gateways). The data formats and protocols used in these integrations may evolve over time.
\end{description}

\subsection{Unlikely Changes} \label{SecUchange}

The module design should be as general as possible. However, a general system is
more complex. Sometimes this complexity is not necessary. Fixing some design
decisions at the system architecture stage can simplify the software design. If
these decision should later need to be changed, then many parts of the design
will potentially need to be modified. Hence, it is not intended that these
decisions will be changed.

\begin{description}
\item[\refstepcounter{ucnum} \uctheucnum \label{ucUse}:] 
  Changes to the primary application of events, registration, tickets, and feedback. The development is focused solely on event management and is not expected to change to accomodate other uses.
\item[\refstepcounter{ucnum} \uctheucnum \label{ucStack}:] 
  Migrating to a different technology stack. The current architecture is designed around an existing framework. A different technology stack would require reimplementing nearly all modules.
\item[\refstepcounter{ucnum} \uctheucnum \label{ucData}:] 
  Transitioning from a centralized database model. The platform relies on centralized data storage for all event management, and applies role-based access to ensure users are only viewing permitted information. Moving to a different model (eg. decentralized) would conflict with the functional requirements, and require rethinking the entire user interaction model.
\end{description}


\section{Module Hierarchy} \label{SecMH}

This section provides an overview of the module design. Modules are summarized
in a hierarchy decomposed by secrets in Table \ref{TblMH}. The modules listed
below, which are leaves in the hierarchy tree, are the modules that will
actually be implemented.

\begin{description}
\item [\refstepcounter{mnum}\mthemnum\label{m1}:] Main System Module

\item [\refstepcounter{mnum}\mthemnum\label{m2}:] User Authentication Module

\item [\refstepcounter{mnum}\mthemnum\label{m3}:] User Authorization Module

\item [\refstepcounter{mnum}\mthemnum\label{m4}:] Form Template Module

\item [\refstepcounter{mnum}\mthemnum\label{m5}:] Form Submission Module

\item [\refstepcounter{mnum}\mthemnum\label{m6}:] Event Management Module

\item [\refstepcounter{mnum}\mthemnum\label{m7}:] Event Notification Module

\item [\refstepcounter{mnum}\mthemnum\label{m8}:] Registration Module

\item [\refstepcounter{mnum}\mthemnum\label{m9}:] Attendance Tracking Module

\item [\refstepcounter{mnum}\mthemnum\label{m10}:] Report Generation Module

\item [\refstepcounter{mnum}\mthemnum\label{m11}:] Analytics Module

\item [\refstepcounter{mnum}\mthemnum\label{m12}:] Database Access Module

\item [\refstepcounter{mnum}\mthemnum\label{m13}:] Audit Module

\end{description}


\begin{table}[H]
\centering
\begin{tabular}{p{0.3\textwidth} p{0.6\textwidth}}
\toprule
\textbf{Level 1} & \textbf{Level 2}\\
\midrule

Hardware-Hiding Module
& No Modules\\
\midrule

\multirow{9}{0.3\textwidth}{Behaviour-Hiding Modules} 
& \mref{m1}: Main System Module\\
& \mref{m2}: User Authentication Module\\
& \mref{m3}: User Authorization Module\\
& \mref{m4}: Form Template Module\\
& \mref{m5}: Form Submission Module\\
& \mref{m6}: Event Management Module\\
& \mref{m7}: Event Notification Module\\
& \mref{m8}: Registration Module\\
& \mref{m9}: Attendance Tracking Module\\
& \mref{m10}: Report Generation Module\\
\midrule

\multirow{3}{0.3\textwidth}{Software-Decision Modules} 
& \mref{m11}: Analytics Module\\
& \mref{m12}: Database Access Module\\
& \mref{m13}: Audit Module\\
\bottomrule

\end{tabular}
\caption{Module Hierarchy}
\label{TblMH}
\end{table}

\section{Connection Between Requirements and Design} \label{SecConnection}

The design of the system is intended to satisfy the requirements developed in
the SRS. In this stage, the system is decomposed into modules. The connection
between requirements and modules is listed in Table~\ref{TblRT}.

\wss{The intention of this section is to document decisions that are made
  ``between'' the requirements and the design.  To satisfy some requirements,
  design decisions need to be made.  Rather than make these decisions implicit,
  they are explicitly recorded here.  For instance, if a program has security
  requirements, a specific design decision may be made to satisfy those
  requirements with a password.}

\section{Module Decomposition} \label{SecMD}

Modules are decomposed according to the principle of ``information hiding''
proposed by \citet{ParnasEtAl1984}. The \emph{Secrets} field in a module
decomposition is a brief statement of the design decision hidden by the
module. The \emph{Services} field specifies \emph{what} the module will do
without documenting \emph{how} to do it. For each module, a suggestion for the
implementing software is given under the \emph{Implemented By} title. If the
entry is \emph{OS}, this means that the module is provided by the operating
system or by standard programming language libraries.  \emph{\progname{}} means the
module will be implemented by the \progname{} software.

Only the leaf modules in the hierarchy have to be implemented. If a dash
(\emph{--}) is shown, this means that the module is not a leaf and will not have
to be implemented.

\subsection{Hardware Hiding Modules}

% \begin{description}
% \item[Secrets:]The data structure and algorithm used to implement the virtual
%   hardware.
% \item[Services:]Serves as a virtual hardware used by the rest of the
%   system. This module provides the interface between the hardware and the
%   software. So, the system can use it to display outputs or to accept inputs.
% \item[Implemented By:] OS
% \end{description}
There are no hardware components for this system.
\subsection{Behaviour-Hiding Module}

% \begin{description}
% \item[Secrets:]The contents of the required behaviours.
% \item[Services:]Includes programs that provide externally visible behaviour of
%   the system as specified in the software requirements specification (SRS)
%   documents. This module serves as a communication layer between the
%   hardware-hiding module and the software decision module. The programs in this
%   module will need to change if there are changes in the SRS.
% \item[Implemented By:] --
% \end{description}

\subsubsection{Main System Module (\mref{m1})}
\textbf{Secrets:} Defines the business logic which allows for the communication between the different modules and with external users\\
\textbf{Services:} Manages and stores the application states and context and serves as the central entrypoint to the system and which modules to communicate with.\\
\textbf{Implemented By:} React and NodeJS\\
\textbf{Module Type:} Abstract Object Module.

\subsubsection{User Authentication Module (\mref{m2})}
\textbf{Secrets:} The internal methods used for verifying user identities within the system.\\
\textbf{Services:} Authenticates users by validating credentials and provides access to the system.\\
\textbf{Implemented By:} Backend JavaScript API.\\
\textbf{Module Type:} Abstract Object Module.

\subsubsection{User Authorization Module (\mref{m3})}
\textbf{Secrets:} Rules defining access levels and user roles. Defines which features are to be accessed by each role.\\
\textbf{Services:} Provides the system with a ruleset on what each roles have access to and limits/grants access to different parts of the system to users based on credentials. \\
\textbf{Implemented By:} Backend JavaScript API.\\
\textbf{Module Type:} Abstract Data Type Module.

\subsubsection{Form Template Module (\mref{m4})}
\textbf{Secrets:} The structure and layout of form templates used in surveys and event creation.\\
\textbf{Services:} Provides reusable blueprints for constructing event or survey forms. Allows for the creation of new forms and editing of preexisting forms.\\
\textbf{Implemented By:} React Frontend, Backend PostgreSQL Database Schemas.\\
\textbf{Module Type:} Abstract Data Type Module.

\subsubsection{Form Submission Module (\mref{m5})}
\textbf{Secrets:} Validation rules of the user input data.\\
\textbf{Services:} Provides mechanism for user to answer and submit forms. Receives and validates user-submitted forms and stores them in database.\\
\textbf{Implemented By:} Backend form processing JavaScript API.\\
\textbf{Module Type:} Abstract Object Module.

\subsubsection{Event Management Module (\mref{m6})}
\textbf{Secrets:} Event data structures and scheduling rules.\\
\textbf{Services:} Enables creation, editing, and cancellation of events by administrators.\\
\textbf{Implemented By:} React Frontend with Backend event processing JavaScript API.\\
\textbf{Module Type:} Abstract Object Module.

\subsubsection{Event Notification Module (\mref{m7})}
\textbf{Secrets:} Notification delivery logic and timing rules.\\
\textbf{Services:} Sends alerts to users regarding new events, updates, or cancellations.\\
\textbf{Implemented By:} Third-Party Messaging Apis.\\
\textbf{Module Type:} Library.

\subsubsection{Registration Module (\mref{m8})}
\textbf{Secrets:} Mapping between users, events, and registration states.\\
\textbf{Services:} Allows users to register, modify, or cancel event participation. Provides validation and confirmation of registration within the event. Allows for admins to view users who have registered for each event and the stage of the process they are in.\\
\textbf{Implemented By:} Backend JavaScript APIs.\\
\textbf{Module Type:} Abstract Data Type Module.

\subsubsection{Attendance Tracking Module (\mref{m9})}
\textbf{Secrets:} Methods for recording attendance and validating entry codes.\\
\textbf{Services:} Tracks event attendance and verifies participant access. Allows for admins for to view \\
\textbf{Implemented By:} Backend JavaScript APIs.\\
\textbf{Module Type:} Abstract Data Object Module.

\subsubsection{Report Generation Module (\mref{m10})}
\textbf{Secrets:} Formatting logic for report generation and provides static data structure of the report.\\
\textbf{Services:} Converts analytics data into exportable human-readable reports.\\
\textbf{Implemented By:} Third Part Apis with JavaScript and React.\\
\textbf{Module Type:} Abstract Data Object Module.

\subsection{Software Decision Module}

% \begin{description}
% \item[Secrets:] The design decision based on mathematical theorems, physical
%   facts, or programming considerations. The secrets of this module are
%   \emph{not} described in the SRS.
% \item[Services:] Includes data structure and algorithms used in the system that
%   do not provide direct interaction with the user. 
%   % Changes in these modules are more likely to be motivated by a desire to
%   % improve performance than by externally imposed changes.
% \item[Implemented By:] --
% \end{description}

\subsubsection{Analytics Module (\mref{m11})}
\textbf{Secrets:} Algorithms for aggregating and calculating statistics.\\
\textbf{Services:} Computes summaries based on the data provided for set statistics and performs data analysis.\\
\textbf{Implemented By:} Javascript Backend Library.\\
\textbf{Module Type:} Library.

\subsubsection{Database Access Module (\mref{m12})}
\textbf{Secrets:} Database schema design and access.\\
\textbf{Services:} Provides database operations to query and insert to database through a unified database layer.\\
\textbf{Implemented By:} Drizzle ORM with JavaScript APIs.
\textbf{Module Type:} Abstract Data Object Module.

\subsubsection{Audit Module (\mref{m13})}
\textbf{Secrets:} Policy and data structure for recording administrative actions.\\
\textbf{Services:} Provides methods to track user and admin activities to ensure traceability and compliance.\\
\textbf{Implemented By:} Backend Javascript API with PostgreSQL Database.
\textbf{Module Type:} Abstract Data Type Module.

\section{Traceability Matrix} \label{SecTM}

This section shows two traceability matrices: between the modules and the
requirements and between the modules and the anticipated changes.

% the table should use mref, the requirements should be named, use something
% like fref
\begin{table}[H]
\centering
\begin{tabular}{p{0.2\textwidth} p{0.6\textwidth}}
\toprule
\textbf{Req.} & \textbf{Modules}\\
\midrule
FR-1 & \mref{m1},\mref{m3}, \mref{m4}\\
FR-2 & \mref{m1},\mref{m9}, \mref{m10}\\
FR-3 & \mref{m1},\mref{m7}, \mref{m5}\\
FR-4 & \mref{m1},\mref{m7}\\
FR-5 & \mref{m1},\mref{m5}\\
FR-6 & \mref{m1},\mref{m11}\\
FR-7 & \mref{m1},\mref{m11}, \mref{m3}, \mref{m4}\\
FR-8 & \mref{m1},\mref{m9}, \mref{m10}\\
FR-9 & \mref{m1},\mref{m3}, \mref{m4}\\
FR-10 & \mref{m1},\mref{m5}\\
FR-11 & \mref{m1}, \mref{m2}\\


\bottomrule
\end{tabular}
\caption{Trace Between Functional Requirements and Modules}
\label{TblRT}
\end{table}

\begin{table}[H]
\centering
\begin{tabular}{p{0.22\textwidth} p{0.68\textwidth}}
\toprule
\textbf{Req.} & \textbf{Modules}\\
\midrule
AR-1 & \mref{m1} \\
AR-2 & \mref{m1} \\
AR-3 & \mref{m1}, \mref{m10}, \mref{m11} \\
SR-1 & \mref{m1}\\
SR-2 & \mref{m1}, \mref{m4}, , \mref{m6}, \mref{m10} \\
\bottomrule
\end{tabular}
\caption{Trace between Look and Feel Requirements and Modules}
\label{TblRT_LF}
\end{table}

\begin{table}[H]
\centering
\begin{tabular}{p{0.20\textwidth} p{0.70\textwidth}}
\toprule
\textbf{Req.} & \textbf{Modules}\\
\midrule
ER-1 & \mref{m4}, \mref{m5}, \mref{m8}  \\
ER-2 & \mref{m1}, \mref{m2} \\
ER-3 & \mref{m1}, \mref{m4}, \mref{m10}, \mref{m11} \\
PI-1 & \mref{m4} \\
PI-2 & \mref{m1}, \mref{m9}, \mref{m11} \\
LR-1 & \mref{m1} \\
LR-2 & \mref{m1}, \mref{m4} \\
UR-1 & \mref{m1} \\
UR-2 & \mref{m1}\\
UR-3 & \mref{m1}, \mref{m12}, \mref{m13} \\
UR-4 & \mref{m1} \\
AC-1 & \mref{m1} \\
\bottomrule
\end{tabular}
\caption{Trace between Usability and Humanity Requirements and Modules}
\label{TblRT_Usability}
\end{table}

\begin{table}[H]
\centering
\begin{tabular}{p{0.20\textwidth} p{0.70\textwidth}}
\toprule
\textbf{Req.} & \textbf{Modules}\\
\midrule
SL-1 & \mref{m1}, \mref{m5}, \mref{m8}, \mref{m12} \\
SL-2 & \mref{m9}, \mref{m11}, \mref{m12} \\
SC-1 & \mref{m1}, \mref{m12}, \mref{m13} \\
PA-1 & \mref{m11} \\
FT-1 & \mref{m1}, \mref{m5} \\
FT-2 & \mref{m1}, \mref{m5}, \mref{m12} \\
FT-3 & \mref{m1}, \mref{m4}, \mref{m12} \\
FT-4 & \mref{m1}, \mref{m6}, \mref{m12} \\
FT-5 & \mref{m11} \\
CR-1 & \mref{m1}, \mref{m12} \\
CR-2 & \mref{m12} \\
SE-1 & \mref{m12} \\
SE-2 & \mref{m1}, \mref{m11}, \mref{m12} \\
LG-1 & \mref{m1} \\
LG-2 & \mref{m1}, \mref{m12}, \mref{m10} \\
\bottomrule
\end{tabular}
\caption{Trace between Performance Requirements and Modules}
\label{TblRT_Perf}
\end{table}

\begin{table}[H]
\centering
\begin{tabular}{p{0.20\textwidth} p{0.70\textwidth}}
\toprule
\textbf{Req.} & \textbf{Modules}\\
\midrule
PE-1 & \\
IR-1 & \mref{m1} \\
IR-2 & \mref{m1} \\
PD-1 & \mref{m1} \\
PD-2 & \mref{m1} \\
RR-1 &  \\
\bottomrule
\end{tabular}
\caption{Trace between Operational \& Environmental Requirements and Modules}
\label{TblRT_OpEnv}
\end{table}

\begin{table}[H]
\centering
\begin{tabular}{p{0.22\textwidth} p{0.68\textwidth}}
\toprule
\textbf{Req.} & \textbf{Modules}\\
\midrule
MT-1 & \\
SU-1 & \mref{m1} \\
AD-1 & \mref{m1} \\
AD-2 & \mref{m1} \\
\bottomrule
\end{tabular}
\caption{Trace between Maintainability \& Support Requirements and Modules}
\label{TblRT_Maintain}
\end{table}

\begin{table}[H]
\centering
\begin{tabular}{p{0.22\textwidth} p{0.68\textwidth}}
\toprule
\textbf{Req.} & \textbf{Modules}\\
\midrule
AC-1 & \mref{m2} \\
AC-2 & \mref{m2}, \mref{m3} \\
IG-1 & \mref{m1}, \mref{m5}, \mref{m12} \\
IG-2 & \mref{m1} \\
PV-1 & \mref{m1}, \mref{m12}, \mref{m13} \\
PV-2 & \mref{m1}, \mref{m2}, \mref{m12} \\
PV-3 & \mref{m12}, \mref{m13} \\
AU-1 & \mref{m13} \\
\bottomrule
\end{tabular}
\caption{Trace between Security Requirements and Modules}
\label{TblRT_Security}
\end{table}

\begin{table}[H]
\centering
\begin{tabular}{p{0.25\textwidth} p{0.65\textwidth}}
\toprule
\textbf{Req.} & \textbf{Modules}\\
\midrule
CL-1 & \mref{m1} \\
\bottomrule
\end{tabular}
\caption{Trace between Cultural Requirements and Modules}
\label{TblRT_Cultural}
\end{table}

\begin{table}[H]
\centering
\begin{tabular}{p{0.25\textwidth} p{0.65\textwidth}}
\toprule
\textbf{Req.} & \textbf{Modules}\\
\midrule
LG-1 & \mref{m12}, \mref{m13} \\
ST-1 & \mref{m1} \\
\bottomrule
\end{tabular}
\caption{Trace between Compliance Requirements and Modules}
\label{TblRT_Compliance}
\end{table}

\begin{table}[H]
\centering
\begin{tabular}{p{0.2\textwidth} p{0.6\textwidth}}
\toprule
\textbf{AC} & \textbf{Modules}\\
\midrule
\acref{acHardware} & \mref{mHH}\\
\acref{acInput} & \mref{mInput}\\
\acref{acParams} & \mref{mParams}\\
\acref{acVerify} & \mref{mVerify}\\
\acref{acOutput} & \mref{mOutput}\\
\acref{acVerifyOut} & \mref{mVerifyOut}\\
\acref{acODEs} & \mref{mODEs}\\
\acref{acEnergy} & \mref{mEnergy}\\
\acref{acControl} & \mref{mControl}\\
\acref{acSeqDS} & \mref{mSeqDS}\\
\acref{acSolver} & \mref{mSolver}\\
\acref{acPlot} & \mref{mPlot}\\
\bottomrule
\end{tabular}
\caption{Trace Between Anticipated Changes and Modules}
\label{TblACT}
\end{table}

\section{Use Hierarchy Between Modules} \label{SecUse}

In this section, the uses hierarchy between modules is
provided. \citet{Parnas1978} said of two programs A and B that A {\em uses} B if
correct execution of B may be necessary for A to complete the task described in
its specification. That is, A {\em uses} B if there exist situations in which
the correct functioning of A depends upon the availability of a correct
implementation of B.  Figure \ref{FigUH} illustrates the use relation between
the modules. It can be seen that the graph is a directed acyclic graph
(DAG). Each level of the hierarchy offers a testable and usable subset of the
system, and modules in the higher level of the hierarchy are essentially simpler
because they use modules from the lower levels.

% \wss{The uses relation is not a data flow diagram.  In the code there will often
% be an import statement in module A when it directly uses module B.  Module B
% provides the services that module A needs.  The code for module A needs to be
% able to see these services (hence the import statement).  Since the uses
% relation is transitive, there is a use relation without an import, but the
% arrows in the diagram typically correspond to the presence of import statement.}

% \wss{If module A uses module B, the arrow is directed from A to B.}

\begin{figure}[H]
\centering
\includegraphics[width=0.9\textwidth]{UsesHierarchy.png}
\caption{Use hierarchy among modules}
\label{FigUH}
\end{figure}

%\section*{References}

\section{User Interfaces}

\wss{Design of user interface for software and hardware.  Attach an appendix if
needed. Drawings, Sketches, Figma}

\section{Design of Communication Protocols}

\wss{If appropriate}

\section{Timeline}

\wss{Schedule of tasks and who is responsible}

\wss{You can point to GitHub if this information is included there}

\bibliographystyle {plainnat}
\bibliography{../../../refs/References}

\newpage{}

\end{document}